\documentclass[12pt,a4paper]{article}
\usepackage[utf8]{inputenc}
\usepackage[spanish]{babel}
\usepackage[T1]{fontenc}
\usepackage{amsmath}
\usepackage{amsthm}
\usepackage{amsfonts}
\usepackage{amssymb}
\usepackage{graphicx}
\usepackage{physics}
\usepackage{mathtools}
\usepackage{hyperref}
\usepackage{xcolor}
\hypersetup{
  colorlinks=false,
  allbordercolors=white
}


\theoremstyle{definition}
\newtheorem{df}{Definición}

\theoremstyle{definition}
\newtheorem{cor}{Corolario}

\theoremstyle{definition}
\newtheorem*{nt}{Notación}

\theoremstyle{remark}
\newtheorem{obs}{Observación}

\theoremstyle{plain}
\newtheorem{prop}{Proposición}

\theoremstyle{plain}
\newtheorem{lema}{Lema}

\theoremstyle{plain}
\newtheorem{teo}{Teorema}

\newcommand{\C}{\mathbb{C}}
\newcommand{\R}{\mathbb{R}}
\newcommand{\N}{\mathbb{N}}
\newcommand{\Z}{\mathbb{Z}}
\newcommand{\Q}{\mathbb{Q}}
\newcommand{\Cont}{\mathcal{C}}


\newcommand{\md}[1]{\left|#1\right|}

\newcommand{\superscriptbefore}[2]{\prescript{#1}{}{#2}}
\newcommand{\subscriptbefore}[2]{\prescript{}{#1}{#2}}

\DeclareMathOperator{\mcm}{mcm}
\DeclareMathOperator{\mcd}{mcd}
\DeclareMathOperator{\car}{car}
\DeclareMathOperator{\Ad}{Ad}
\DeclareMathOperator{\Biad}{Biad}
\DeclareMathOperator{\Ring}{Ring}
\DeclareMathOperator{\Ann}{Ann}
\DeclareMathOperator{\ann}{ann}
\DeclareMathOperator{\id}{id}
\DeclareMathOperator{\End}{End}
\DeclareMathOperator{\Map}{Map}
\DeclareMathOperator{\Soc}{Soc}
\DeclareMathOperator{\minpol}{minpol}

\title{Álgebra Abstracta}
\date{}

\begin{document}

\begin{titlepage}

	\centering
	\scshape
	\vspace*{\baselineskip}
	\rule{\textwidth}{1.6pt}\vspace*{-\baselineskip}\vspace*{2pt}
	\rule{\textwidth}{0.4pt}
	\vspace{0.75\baselineskip}
	{\Huge Álgebra Abstracta\\}
	\vspace{3cm}
	{\LARGE José Antonio de la Rosa Cubero}
	
\end{titlepage}


\tableofcontents
\newpage

\section{Introducción}
\subsection{Generalidades sobre anillos}
\begin{df}[Anillo]
  Sea \(A\) un conjunto en el que existen dos operaciones
  \(+,\cdot:A\times A\longrightarrow A\) tales que:
  \begin{enumerate}
    \item \((A, +,0)\) es un grupo aditivo (conmutativo):
      \begin{itemize}
        \item \((a+b)+c=a+(b+c)\) para todos \(a,b,c\in A\).
        \item \(a+b=b+a\) para todos \(a,b\in A\).
        \item \(a+0=a\) para todo \(a\in A\).
        \item Para todo \(a\in A\) existe un \(-a\in A\)
          tal que \(-a+a=0\).
      \end{itemize}
    \item \((A, \cdot, 1)\) es un monoide:
      \begin{itemize}
        \item \((ab)c=a(bc)\) para todos \(a,b,c\in A\).
        \item \(a\cdot 1=1\cdot a=a\) para todo \(a\in A\).
      \end{itemize}
    \item Se cumplen las siguientes propiedades distributivas:
      \begin{itemize}
        \item \((a+b)c=ac+bc\) para todos \(a,b,c\in A\).
        \item \(a(b+c)=ab+ac\) para todos \(a,b,c\in A\).
      \end{itemize}
     \end{enumerate}
  \end{df}

  \begin{df}[Idelaes]
    Sea \(A\) un anillo. \(I\subset A\) se dice ideal si cumple las
    siguientes propiedades:
    \begin{itemize}
      \item \(I\) es un subgrupo aditivo de \(A\) (es decir,
        \(I\) es un conjunto no vacío que cumple
        \(b-a\in I\) para todo \(a, b\in I\)).
      \item \(ax, xa\in I\) para todo \(a\in I\) y \(x \in A\).
    \end{itemize}
  \end{df}

  \begin{teo}[Teorema de Isomorfía]
    Sea \(f:A\longrightarrow B\) un homomorfismo de anillos. Entonces:
    \begin{itemize}
      \item \(\ker f\) es un ideal de \(A\),
      \item \(\Im f\) es un subanillo de \(B\),
      \item Si \(I\subset \ker f\) es un ideal de \(A\), entonces
        existe un único homomorfismo de anillos tal que
        \(\tilde{f}:A/I\longrightarrow B\) tal que \(\tilde{f}(a+I)=f(a)\).
      \item El homomorfismo anterior es inyectivo si y solo si
        \(I=\ker f\).
      \item El homomorfismo anterior es sobreyectivo si y solo si lo era
        \(f\).
    \end{itemize}
  \end{teo}


\begin{df}[Homomorfismo de anillos]
  \(A,B\) anillos. Se dice que \(f:A\longrightarrow B\) se dice un
  (homo)morfismo de anillos si para todos \(a,a'\in A\) se tiene:
  \begin{enumerate}
    \item \(f(a+a')=f(a)+f(a')\)
    \item \(f(aa')=f(a)f(a')\)
    \item \(f(1)=1\)
  \end{enumerate}
\end{df}

La suma de ideales es un ideal.

\begin{df}[Ideales coprimos]
  Dos ideales \(I, J\subset A\) se dirán primos entre sí
  o coprimos si \(I+J=A\).

  Equivalentemente, existen \(x\in I\), \(y\in J\) tales que
  \(1=x+y\).
\end{df}

La motivación de la definición anterior reside en la identidad
de Bezout, que estamos generalizando.

\begin{lema}
  Sean \(I, J, K\) ideales de \(A\),
  \(I+J=I+K=A\) si y solo si \(I+(J\cap K)=I+J\cap K=A\).

  Es decir, son coprimos entre sí si y solo si uno es coprimo
  con la intersección de los otros dos.
\end{lema}
\begin{proof}
  \[1=x+y=x'+z\]
  con \(x,x'\in I\), \(I\in J\), \(z\in K\).
  \[1=x+y=x+y1=x+y(x'+z)=x+yx'+yz\]
  \(x+yx'\in I\), y \(yz\in J\cap K\).

  Para el recíproco, \(A\supseteq I+J\supseteq I+J\cap K=A\),
  luego \(A=I+J\).
\end{proof}

\begin{lema}
  Sean \(I_1,\ldots, I_t\) ideales de \(A\).
  \(I_1\cap I_i=A\) si y solo si
  \(I_1+\bigcap_{i=2}^t I_i=A\).
\end{lema}
\begin{proof}
  Para \(t=2\) es trivial.

  Supongamos cierto \(I_1\cap I_i=A\implies
  I_1+\bigcap_{i=2}^t I_i=A\) para \(t\), veamos para \(t+1\).

  Llamo \(I=I\), \(J=\bigcap_{i=2}^t I_i\), \(K=I_{t+1}\).
  Por hipótesis de inducción \(I+J=A\) y \(I+K=A\) por ser coprimos
  (hipótesis del lema). Por el lema anterior tenemos:
  \[
    I+J\cap K=I_1+I_{t+1}\cap \bigcap_{i=2}^{t}
    I_i =I_1 + \bigcap_{i=2}^{t+1} I_i
  \]

  La otra implicación es muy sencilla.
\end{proof}

  Hipótesis de trabajo para el teorema chino del resto:
  \begin{enumerate}
    \item \(A\) un anillo.
    \item \(A_1,\ldots,A_t\) anillos.
    \item \(f_i:A\longrightarrow A_i\) un homomorfismo de anillos
      para cada \(i\in\{1,\ldots,t\}\).
    \item \(\Im f_i\subseteq A_i\) es un subanillo.
    \item A \(\Im f_1\times\cdots\times\Im f_t\) se le llama el anillo
      producto.
    \item Definimos \(f:A\longrightarrow\Im f_1\times\cdots\times\Im f_t\),
      \(f(x)=(f_1(x),\ldots,f_t(x))\) para cada \(x\in A\).
    \item Tenemos que \(f\) es un homomorfismo de anillos, cuyo núcleo es la
      intersección de todos los núcleos. Llamaremos \(I=\ker f\).
      \(x\in A\), \(x\in\ker f\) si y solo si \(f_i(x)=0\) para todo \(i\),
      es decir, \(x\in\bigcap_{i=0}^t \ker f_i\).
    \item  Además, existe \(\tilde{f}:A/I
      \longrightarrow\Im f_1\times\cdots\times\Im f_t\),
      con \(x+I\mapsto f(x)\).
    \item Cada \(\ker f_i\) es coprimo con cualquier \(\ker f_j\)
      para \(j\neq i\).
    \item Llamamos \(I_i=\ker f_i\).
  \end{enumerate}

\begin{teo}[Teorema Chino del Resto]
  \(\tilde{f}\) es isomorfismo si y solo si
  \(I_i+I_j=A\) para todo \(i\neq j\).
\end{teo}
\begin{proof}
  Veamos primero la implicación a la derecha.

  Vamos a suponer \(\tilde{f}\) sobreyectiva, es decir, que \(f\) lo es.
  Veamos que todos los \(I_i\) son coprimos entre sí.

  Dado \(i\) tomamos \(x\in A\) tal que \(f_i(x)=1\) y
  \(f_i(x)=0\) para todo \(j\neq i\).

  Observemos que \(x-1\in I_i\), \(x\bigcap_{j\neq i} I_j\)
  \[
    1=1-x+x\in I_i+\bigcap_{j\neq i} I_j
  \]

  Por tanto, \(I_i+\bigcap I_j =A\) y entonces por el lema anterior
  \(I_i+I_j=A\).

  Veamos el recíproco.
  Suponemos que \(I_i+I_j=A\) para cualquier \(i\neq j\).

  Tomamos \((f(b_1),\ldots,f(b_t))\in I_1\times\cdots\times I_t\).

  Para cada \(i\), tomamos
  \(
  1=a_i+p_i\) con \(a_i\in I_i\) y \(p_i\in\bigcap I_j\).

  Tomamos \(x=\sum_{i=1}^t b_i p_i\).

  \[
    f_j(x)=\sum_{k=1}^t f_j(b_k) f_j(p_k)=f_j(b_j)f_j(p_j)=f_j(b_j(1-a_j))
    =f_j(b_j)-f_j(b_j)f_j(a_j)=f_j(b_j)
  \]
  porque \(f_j(p_k)=0\) si \(k\neq j\) y \(a_j\in\ker f_j\).

\end{proof}


\begin{obs}
  Para anillos conmutativos denotamos
  \[
    \langle a\rangle=\{ba:b\in A\}
  \]
  el ideal generado por \(a\).
\end{obs}

Vamos a hacer un ejemplo, aplicando el teorema anterior.

\subsubsection{Interpolación}         

Tomamos \(A=K[x]\), un anillo de polinomios con coeficientes en un
cuerpo \(K\).

Sea \(A_i = K\) con \(i\in\{1,\ldots,t\}\).
Tomamos \(\alpha_i\in K\) para cada \(i\) y definimos
\(\xi_i:K[x]\longrightarrow K\), \(\xi(g)=g(\alpha_i)\),
para cada \(g\in K[x]\) y es un homeomorfismo de anillos.

\(\Im X_i=K\) y \(\xi:K[x]\longrightarrow K\times
\cdots K=K^t\).

\(\ker \xi_i=\langle x-\alpha_i\rangle\) que es ideal de un anillo de
polinomios, luego principal. Está generado por el polinomio de grado menor,
como las constantes no pueden anular a \(\xi_i\), tiene que estar generado
por ese, que es de grado uno.

\[
  I=\bigcap_{i=1}^t\langle x-\alpha_i\rangle =\langle p(x)\rangle
\]
donde \(p(x)=\mcm\{x-\alpha_i: i\in\{1,\ldots, t\}\}\).


El teorema chino del resto nos asegura que \(\tilde{\xi}:
K[x]/\langle p(x)\rangle\longrightarrow K^t\) es un isomorfismo si y solo si
\(\mcd\{x-\alpha_i, x-\alpha_j\}\) para todo \(j\neq i\), es decir,
si \(\alpha_i\neq \alpha_j\).

Lo que estamos viendo es que para cualquier tupla
\((y_1,\ldots, y_t)\in K^t\), existe un \(g\in K[x]\) tal que
\(g(\alpha_i)=y_i\), si y solo si \(\alpha_i\neq\alpha_j\). En tal
caso, \(p(x)=\prod_{i=1}^t(x-\alpha_i)\).

Existe un único representante \(g\in K[x]\) tal que \(g(\alpha_i)=y_i\)
de grado menor que \(t\), siempre que
\(p(x)=\prod_{i=1}^t(x-\alpha_i)\).

\(\alpha_1,\ldots,\alpha_t\in K\) distintos dos a dos
\[
  \tilde{\xi}:K[x]/\langle p(x)\rangle\longrightarrow K^t
\]
es un isomorfismo de anillos.

\(K[x]/\langle p(x)\rangle\) es un espacio vectorial cociente.

\(\tilde{\xi}\) es también un isomorfismo entre espacios vectoriales.

\[
  \tilde{\xi}(\alpha(g+p))=\tilde{\xi}(\alpha g+p)=\tilde{\xi}((\alpha
  + p)(g +p))=\]\[\tilde{\xi}(\alpha + p)\tilde{\xi}(g+p)=
  (\alpha,\ldots, \alpha)(g(\alpha_1),\ldots g(\alpha_t))=
  \alpha\tilde{\xi}(g+p)
\]

Sea \(\{1+p, x+p, x^2+p,\ldots, x^{t-1} +p\}\)
\(K\)-base de \(K[x]/\langle p(x)\rangle\).
Notamos:
\[ 1 = 1+p\]
\[ x = x+p\]

Sea \(\{e_1,\ldots, e_n\}\) es la base canónica de \(K^t\).
Nuestro objetivo es calcular sus preimagenes por \(\xi\), en concreto
un polinomio de grado menor que \(t\).

\[
  g_i(x)=\prod_{j\neq i}(x-\alpha_j)
\]

\[
  L_i(x)=\frac{g_i(x)}{g(\alpha_i)}=\prod_{j\neq i}\frac{x-x_j}{x_i-x_j}
\]

que vale 0 en \(\alpha_j\) para cualquier \(j\)
salvo en \(\alpha_i\) que vale 1.

Tenemos que
\[
  g(x)=\sum_{i=1}^t y_i L_i(x)
\]
satisface que \(g(\alpha_i)=y_i\).


Finalmente vamos a ver que la matriz de \(\tilde{\xi}\) en las bases
consideradas es:
\[
  \begin{pmatrix}
    1&\cdots& 1\\
    \alpha_1 &\cdots&\alpha_t\\
    \cdots&\cdots&\cdots\\
    \alpha_1^t&\cdots&\alpha_t^t
  \end{pmatrix}
\]

\subsubsection{Transformada discreta de Fourier}

Ahora vamos a reindexar. En lugar de usar \(1,\ldots, t\)
vamos a tomar los índices \(1,\ldots, n-1\).

Vamos a suponer que el cuerpo \(K\)
contiene una raíz primitiva de 1, o sea,
existe un \(\omega\in K\) tal que \(\omega^n = 1\) y
\(1, \omega, \omega^2,\ldots, \omega^{n-1}\) son distintos.

Seguro que \(\car K \not\| n\) ya que \(1, \omega, \omega^2,\ldots,
\omega^{n-1}\) son las raíces de \(x^n-1\) y son distintas.

Vamos a interpolar las raíces de la unidad.

Tomo \(\alpha_j =\omega^j\), \(j\in\{0,\ldots, n\}\) y
\[M=A_\omega=
\begin{pmatrix}
  1&1&\cdots1\\
  \omega^0&\omega^1&\cdots\omega^{n-1}\\
  {(\omega^0)}^2&{(\omega^1)}^2&\cdots{(\omega^{n-1})}^2\\
  \cdots&\cdots&\cdots
\end{pmatrix}= (\omega^{ij})
\]


Tenemos que \(x^n-1=(x-1)(x^{n-1}+\cdots+x+1)\)
y evaluando en \(\omega^{j}\) obtenemos
\[
  \omega^{(n-1)j}+\cdots+\omega^j+1=0
\]

Entonces \(\sum_{k=0}^{n-1}\omega^{ik}=0\) para \(0<i<n\).

\[
  \begin{pmatrix}
    \omega^{i}&
    \omega^{2i}&
    \cdots&
    \omega^{(n-1)i}
  \end{pmatrix}
  \begin{pmatrix}
    \omega^{-j}\\
    \omega^{-2j}\\
    \cdots\\
    \omega^{-(n-1)j}
  \end{pmatrix}
  =\sum_{k=0}^{n-1}\omega^{k(i-j)}=0
\]

Tenemos entonces que \(A_\omega A_{\omega^{-1}}^T=nI\),
es decir, \(A^{-1}_\omega=\frac{1}{n}A_{\omega^{-1}}^T\).

\(\tilde{\xi}:K[x]/\langle x^n-1\rangle\longrightarrow K^n\),
con \(\xi^{-1}(y)\) es el polinomio interpolador.

Tenemos unos datos \((y_0,\ldots, y_{n-1})\in K^n\). El polinomio
interpolador de esos datos en los nodos \(1,\omega,\ldots,\omega^{n-1}\)
viene dado por
\[
  \hat{y} =\sum_{j=0}^{n-1}\hat{y_j}x^j
\]
donde \(\hat{y}=y\frac{1}{n}A^T_{\omega^{-1}}\).

Explicitamente, se calcula que los coeficientes quedan:
\[
  y_j=\frac{1}{n}\sum_{k=0}^{n-1}y_k\omega^{-jk}
\]

Tomamos \(K=\C\). \(\omega = e^{i2\pi/n}\):
\[
  y_j=\frac{1}{n}\sum_{k=0}^{n-1}y_k\omega^{-i2\pi jk/n}
\]
que es la transformada de Fourier de \(y\).

¿Qué interpretación le damos? Supongamos una función periódica de periodo
\(2\pi\), \(f:[0,2\pi]\longrightarrow\C\) con \(f(0)=f(2\pi)\).
Dividimos el intervalo en \(n\) partes iguales, una muestra:
\(y_j=f(\frac{2\pi j}{n})\) con \(j=0,\ldots,n-1\).

Tomomamos \(g:[0,2\pi]\longrightarrow\C\) con
\(g(t)=\sum_{j=0}^{n-1}\hat{y_j}e^{ijt}\).

Tenemos entonces que \(g(\frac{2\pi l}{n})=
\sum_{l=0}^{n-1}\hat{y_j}e^{i2\pi lj/n} = y_l=f(\frac{2\pi j}{n})\)

A los \(\hat{y}\) también se le llama el espectro de \(y\).

\section{Módulos}

\begin{df}
  Sean \(M\), \(N\) grupos aditivos:
  \[
    \Ad(M,N)=\{f:M\longrightarrow N| f\textrm{ homomorfismo de grupos}\}
  \]
\end{df}

El conjunto anterior es no vacío porque \(0\in\Ad(M,N)\).
\(\Ad(M,N)\) es un grupo aditivo con la suma:
\[
  (f+g)(m):=f(m)+g(m)\hspace{1cm} \forall m\in M
\]


\begin{df}[Anillo de endomorfismo de \(M\)]
  Definimos directamente \(\End(M):= \Ad(M,M)\).
\end{df}

\begin{prop}
  \((\End(M),+,0,\circ,\id) \) es un anillo.
\end{prop}
\begin{proof}
  Se comprueba que es cerrado para composición. Es obvio que la
  composición es asociativa y tiene como elemento neutro la identidad.

  Finalmente se ve que se cumplen las propiedades distributivas, que
  se siguen de que son homomorfismos.
\end{proof}


\begin{obs}
  Consideramos el grupo \(\{0\}\), es el anillo \(\{0\}\) (anillo
  cero o trivial).

  Si \(M\neq \{0\}\), entonces \(\End(M)\) no es trivial.
\end{obs}

\begin{df}[Módulo]
  Sea \(M\) un grupo aditivo y \(A\) un anillo. Una estructura de
  \(A\)-módulo sobre \(M\) es un homomorfismo de anillos
  \(\rho: A\longrightarrow\End(M)\).
\end{df}

Ejemplo: los números enteros. \(M\) grupo aditivo, \(A=\Z\).
Existe un único \(\chi:\Z\longrightarrow\End(M)\) determinado
por \(\chi(1)=\id_M\), es decir, una única estructura de
\(\Z\)-módulo sobre \(M\) (y su núcleo te da la
característica del anillo).

Ejemplo: cuerpos. Sea \(K\) un cuerpo.
Si \(V\) es un \(K\)-espacio vectorial, definimos \(\rho:K\longrightarrow
\End(V)\), tomamos \(\rho(\alpha):V\longrightarrow V\)
cumpliendo \(\rho(\alpha)(v)=\alpha v\). Trivialmente se cumple que
\(\rho\) es un homomorfismo por la estructura de espacio vectorial de \(V\).
Con lo cual tenemos una estructura de \(K\)-módulo sobre \(V\).
Se puede demostrar el recíproco trivialmente.

\begin{obs}
  Sean  \(X, Y, Z\) conjuntos. \(\Map(X,Y)\) es el conjunto de
  aplicaciones de \(X\) en \(Y\).

  Entonces:
  \[
    \psi:\Map(X\times Y, Z)\longrightarrow\Map(X,\Map(Y,Z))
  \]
  es una biyección dada por \(\psi(f)(x)(y):=f(x,y)\) y
  \(\psi^{-1}(g)(x,y):=g(x)(y)\).
\end{obs}

Ejercicio: comprobar que \(\psi^{-1}\) es realmente la inversa de
\(\psi\).

\begin{obs}
  Sean \(M, N, L\) grupos aditivos.
  \[
    \psi:\Biad(M\times N, L)\longrightarrow\Ad(M,\Ad(N,L))
  \]
  donde \(b\in\Biad(M\times N, L)\) si \(b\) es biaditiva:
  \[
    b(m+m',n)=b(m,n)+b(m',n)
  \]\[
    b(m,n+n')=b(m,n)+b(m,n')
  \]
\end{obs}

Ejercicio, demostrar que la aplicación \(\psi\) es una biyección.

\begin{teo}[Caracterización de módulos]
  Sea \(A\) anillo, \(M\) un grupo aditivo. Sea \(\Ring(A,\End(M))\),
  llamamos \(A\)-módulo a la imagen por \(\psi\) de ese conjunto.
\end{teo}

\begin{df}
  \[
    \Ring(R,S)=\{\phi:R\longrightarrow S, \phi \textrm{ es homomorfismo
    de anillos}\}
  \]
\end{df}


\begin{prop}
  Dados un grupo aditivo \(M\) y un anillo \(A\), se tiene una
  correspondencia biyectiva entre:
  \begin{enumerate}
    \item Homomorfismos de anillos \(\rho:A\longrightarrow\End(M)\)
    \item Las aplicaciones \(A\times M\longrightarrow M\)
      que satisfacen:
      \begin{itemize}
        \item \((a+a')m=am+a'm\)
        \item \(a(m+m')=am+am'\)
        \item \((aa')m=a(a'm)\)
        \item \(1\cdot m=m\)
      \end{itemize}
  \end{enumerate}
\end{prop}

\begin{proof}
  Tomamos la biyección \(\psi^{-1}:\Map(A,\Map(M,M))\longrightarrow
  \Map(A\times M,M)\).
  Tomamos \(\rho\in\Ring(A,\End(M))\), su imagen por la biyección,
  \(\psi^{-1}(\rho)\)
  son las aplicaciones que satisfacen justo las propiedades
  anteriores.

  Llamamos a \(\psi^{-1}(\rho)(a,m)=a\cdot m\). Tenemos que
  \(\psi^{-1}(\rho)(a,m)=\rho(a)(m)\). Entonces
  \(a\cdot m=\rho(a)(m)\).

  Comprobamos la tercera propiedad como ejemplo:

  Dados \(a, a'\in A\) y \(m\in M\):
  \[
    (aa')m=\rho(aa')(m)=(\rho(a)\circ\rho(a'))(m)
    =\rho(a)(\rho(a')(m))=\rho(a)(a'm)=a(a'm)
  \]

  De forma análoga se demuestran el resto de propiedades.

  Esta correspondencia responde a la fórmula \(am=\rho(a)(m)\).
\end{proof}

Un \(A\)-módulo lo veré de cualquiera de las maneras anteriores, que
ya hemos visto que son equivalentes, según su conveniencia.

Ejemplo, si \(K\) es un cuerpo, un \(K\)-módulo es esencialmente
un \(K\) espacio vectorial.

Otro ejemplo, el \(A\)-módulo regular. \(A\) es un \(A\)-módulo, vía
\(\lambda:A\longrightarrow(A)\) que lleva cada
\(a\) a \(\lambda(a)(a'):=aa'\). La demostración es sencilla
usando la segunda definición.

\begin{prop}[Restricción de escalares]
  Sea \(\phi:\R\longrightarrow S\) homomorfismo de anillos. Si \(M\)
  es un \(S\)-módulo, vía un homomorfismo de anillos \(\rho:
  S\longrightarrow\End(M)\), tenemos que \(M\) es un
  \(R\)-módulo vía \(\rho\circ\phi\).

  Equivalentemente, si \(r\in R\) y \(m\in M\), definimos
  \[
    rm=(\rho\circ\phi)(r)(m)=\rho(\phi(r))(m)=\phi(r)m
  \]
\end{prop}


\subsection{\(K[x]\)-módulos con \(K\) cuerpo}

Tenemos \(K[x]\)-módulo \(M\). O sea, \(M\) es un grupo aditivo
y \(\rho: K[x]\longrightarrow\End(M)\) es un homomorfismo de anillos.

\(K\) se puede ver como subanillo de \(K[x]\), aplicando la
restricción de escalares aplicada a la aplicación inclusión,
\(M\) es un \(K\)-espacio vectorial.

Veamos que ocurre con la indeterminada. \(\rho(x)\in\End(M)\).

Veamos que es un endomorfismo de espacios vectoriales:
\[
  \rho(x)(km)=x\cdot (km)=x\cdot(k\cdot m)=(xk)\cdot m
  =kx\cdot m=k(xm)=k\rho(x)(m)
\]

Así que \(\rho(x)\) es \(K\)-lineal.

Si \(p=\sum_i p_i x^i\in K[x]\), tenemos que
\[
  pm=\rho(p)(m)=\sum_i p_i {\rho(x)}^i(m)
\]

\begin{prop}
  Si tengo un \(K\)-espacio vectorial \(V\) y una aplicación
  lineal \(T:V\longrightarrow V\), podemos definir para \(p\in K[x]\)
  y \(v\in V\) el operador
  \[
    pv:=p(T)(v)=\sum_i p_i T^i(v)
  \]
  resulta que \(V\) es un \(K[x]\)-módulo.
\end{prop}

Ejemplo, \(\Cont^\infty(\R)\) con \(T=\frac{d}{dt}\)
es un \(\R[x]\)-módulo.

\begin{obs}
  \(\Cont^\infty(\R)\) dotado de estructura de \(\R[x]\)-módulo
  a través del endomorfismo lineal \(T=\frac{d}{dt}\) es un ejemplo
  ilustrativo en el siguiente sentido.

  Tomemos \(\sin\), \(x\sin t=T(\sin t)=\cos t\)
  \(x^2\sin t= -\sin t\) con lo que
  \[
    (x^2+1)\sin t=0
  \]
  es decir, en un \(A\)-módulo \(M\) puede pasar que \(a m=0\)
  \(a\neq 0\), \(m\neq 0\).
\end{obs}

Ejemplo en el \(\Z\)-módulo \(\Z_4\) tenemos que
\(2\cdot \bar{2}=\bar{0}\).

\subsubsection{Módulos abstractos}

Sea \(A\) un anillo, \(M_A\) un \(A\)-módulo,
entonces si tenemos
un homomorfismo de anillos \(\varphi:A\longrightarrow\End(M)\)
cuyo núcleo es un ideal de \(A\).

Aplicando el primer teorema de isomorfía, tenemos:
\[
  A/\ker\varphi\longrightarrow\Im\varphi\subseteq\End(M)
\]
y entonces \(M\) es un \(A/\ker\varphi\)-módulo.
De hecho \((a+\ker\varphi)m=\varphi(m)\).

\[
  \ker\varphi=\{a\in A: am=0\}=\Ann_A(M)
\]
se le llama el anulador de \(M\).

Tenemos que \(M_A\) entonces \(M_{A/\Ann_A(M)}\)


\begin{proof}
  Veamos que 1 implica 2. Tenemos que \(0=n_1+\cdots+n_t\),
  si despejamos, \(n_i=-\sum_{j\neq i} n_j\in N_i\cap\left(
  \sum_{j\neq i} N_j\right)=\{0\}\).

  Veamos que 2 implica 3. Si \(n=\sum n_i=\sum n_i'\),
  entonces \(0=\sum(n_i-n_i')\) lo que implica que
  \(n_i=n_i'\).

  Finalmente, tomando \(n\in N_i\cap\left(
  \sum_{j\neq i} N_j\right)\), es decir,
  \(n=\sum_{j\neq i}n_j\) con lo que
  \(0=n-\sum_{j\neq i} n_j\) y como las descomposiciones
  son únicas, \(n=0\).
\end{proof}

\begin{df}[Suma interna]
  Si \(M=N_1+\cdots+ N_t\) tales que satisfacen una de las condiciones
  equivalentes anteriores, diremos que \(M\) es la suma directa interna
  y usaremos la notación \(M=N_1\dot{+}\cdots\dot{+} N_t\).
\end{df}

\begin{df}
  Si \(\{N_1,\ldots, N_t\}\) verifican las condiciones equivalentes
  anteriores y \(N_i\neq \{0\}\), se dice que el conjunto
  \(\{N_1,\ldots, N_t\}\) es una
  familia independiente.
\end{df}

Ejemplo: \(\Z_6\) es un \(\Z\) módulo.
\[
  \Z_6=\{0,1,2,3,4,5,6\}
\]
Tomamos
\[
  N_1=\{0,3\}
\]
y
\[
  N_2=\{0,2,4\}
\]

Tenemos que \(N_1, N_2\) es una familia independiente. Además es obvio que:
\[
  N_1\dot{+}N_2=\Z_6
\]
ya que tienen como intersección \(\{0\}\) y su suma es el total.


\subsection{Módulos acotados sobre un DIP}

\begin{df}[Módulo acotado sobre un DIP]
  Sea \(A\) un dominio de ideales principales, \(\subscriptbefore{A}{M}\)
  un módulo, \(\Ann_A(M)=\langle\mu\rangle\) para cierto \(\mu\in A\).

  Si \(\mu\neq 0\), diré que \(M\) es acotado.
\end{df}

Supongamos que \(\subscriptbefore{A}{M}\) es acotado y \(\mu\not\in
\mathcal{U}(A)\), ya que si \(\mu\in\mathcal(A)\) entonces \(M=\{0\}\).

Si \(\mu=p_1^{e_1}\cdots p_t^{e_t}\), posible porque todo DIP es un
dominio de factorización única (DFU), con \(p_i\in A\) irreducible
y \(e_i>0\).

\begin{prop}[Descomposición primaria del módulo]
  Tomamos \(q_i=\frac{\mu}{p_i^{e_i}}\in A\).

  Llamamos \(M_i=\{q_i m:m \in M\}\subseteq M\). Veamos
  que \(M_i\in\mathcal{L}(\subscriptbefore{A}{M})=\{\textrm{submódulos
  de } \subscriptbefore{A}{M}\}\).

  Queremos que \(M=M_1\dot{+}\cdots \dot{+}M_t\), con \(t>1\) para
  evitar trivialidades. En ese caso, \(\mcd\{q_1,\ldots,q_t\}=1\),
  donde se ha usado que estamos en un DFU.\@

  Por la identidad de Bezout (válida porque estamos en un DIP),
  tenemos que \(1=\sum_{i=1}^t q_i a_i\), para ciertos \(q_i\in A\).
  Para en \(m\in M\), \(M=1\cdot m=\sum_i q_i a_i m\), luego
  \(M=M_1+\cdots+ M_t\).

  Vamos a ver que la suma es directa.
  \(q_i q_j\in\langle\mu\rangle\) si \(i\neq j\). Eso significa que si
  \(m\in M_i\) y entonces \(q_j m= 0\) si \(i\neq j\).
  Por tanto \(M_i=\{m\in N: m=q_i a_i m\}\).

  Si \(0=\sum_{i=1}^t\) con \(m_i\in M_i\), entonces
  \[
    0=q_j a_j 0=m_j
  \]
  y por tanto \(M=M_1\dot{+}\cdots\dot{+} M_t\).
\end{prop}

\begin{df}[Componentes primarias]
  Tenemos que los \(M_i\) se llaman componentes primarias.
\end{df}
\begin{prop}
  \[
    M_i=\{m\in M: p_i^{e_i} m=0\}
  \]

  Así, \(\langle\mu\rangle=\Ann_A(M)=\bigcap_{i=1}^t\Ann_A(M_i)
  \supseteq\bigcap_{i=1}^t\langle p_i^{e_i}\rangle=\langle\mu\rangle\)
\end{prop}

Ejercicio: Obtener la descomposición primaria usando \(\dot{+}\) de
\(\Z_{8000}\).

Ejemplo: \(T\) endomorfismo \(K\)-lineal. \(V=\subscriptbefore{K[x]}{V}\).

Un \(W\) es un submódulo de \(V\) es un subespacio vectorial tal que
\(T(W)\subseteq W\), es decir, \(W\) es \(T\) invariante.

Si \(\Ann_{K[x]}(V)\neq\{0\}\), tomo \(\mu(x)\in K[x]\), el polinomio
mínimo de \(T\). Es decir, \(\Ann_{K[x]}(V)=\langle\mu(x)\rangle\).

\[
  \mu=p_1^{e_1}\cdots p_t^{e_t}
\]

Entonces la descomposición primaria de \(V\) es \(V=V_1\dot{+}
\cdots\dot{+}V_t\) con
\[
  V_i=\{v\in V:p_i(x)v=0\}
\]

Caso particular: \(\dim(V)<\infty\) y que \(\mu(x)=(x-\alpha_1)\cdots
(x-\alpha_t)\) con \(\alpha_i\neq\alpha_j\).
\[
  V_i=\{v\in V:(x-\alpha_i)v=\{v\in V:T(v)=\alpha_i v\}
\]
es decir, el subespacio propio asociado al valor propio \(\alpha_i\).

Si el polinomio factoriza como producto de polinomios de grado 1 distintos,
\(T\) es diagonalizable.
Veremos en el futuro que el polinomio mínimo divide siempre al polinomio
característico.

¿Cómo se calcula el polinomio mínimo de un endomorfismo lineal?

Ejercicio: Sea \(V\) un espacio vectorial real euclídeo (con producto
escalar). Sea \(T:V\longrightarrow V\) una isometría.
Se pide demostrar que si \(W\) es un subespacio \(T\) invariante de \(V\),
entonces su ortogonal \(W^\perp\) es también \(T\) invariante.
Entonces \(V=W\dot{+}W^\perp\). Se usa inducción. Como consecuencia, usando
el teorema fundamental del álgebra, deducir que \(V\) admite una base
ortonormal con respecto de la cual la matriz de \(T\) es diagonal por
bloques, con bloques de dimensión 1 o 2. ¿Qué aspecto tienen dichos
bloques? Hay que ver que uno de los dos subespacios invariantes tienen
dimensión 1 o 2.


\subsection{Homomorfismos de módulos}

\begin{df}[Módulo cociente o factor]
  Sea \(\subscriptbefore{A}{M}\) y \(L\in\mathcal{L}(M)\).
  Consideramos \(M/L\) grupo aditivo y se define la acción:
  \[
    a(m+L):=am+L
  \]
  \(M/L\) es un módulo.
\end{df}

\begin{df}[Homomorfismo de módulos]
  Se dice que
  \(f:\subscriptbefore{A}{M}\longrightarrow \subscriptbefore{A}{N}\)
  es un homomorfismo de módulos si respeta sumas y productos.
\end{df}

\begin{df}[Proyección canónica]
  Es la aplicación \(\pi:M\longrightarrow M/L\) dada por
  \(\pi(m)=m+L\) es un homomorfismo de módulos.
\end{df}

\begin{teo}[Teorema de isomorfía para módulos]
  \(f:M\longrightarrow N\) un homorfismo de \(A\)-módulos. Entonces
  el núcleo \(\ker f\in\mathcal{L}(\subscriptbefore{A}{M})\) y
  \(\Im f\in\mathcal{L}(N)\). Para cada \(L\in\mathcal{L}
  (\subscriptbefore{A}{M})\) tal que \(L\subseteq \ker f\) existe
  un único homomorfismo de módulos \(\tilde{f}:M/L\longrightarrow N\)
  tal que \(\tilde{f}\circ\pi=f\). Finalmente, \(\tilde{f}\) es
  inyectiva si y solo si \(L=\ker f\), en cuyo caso, \(\tilde{f}\)
  da un isomorfismo de \(A\)-módulos
  \(M/\ker f\cong \Im f\).
\end{teo}

Ejemplo \(\subscriptbefore{A}{M}\), definimos \(f:A\longrightarrow M\)
dada por:
\[
  f(a)=am\hspace{1cm} \forall a \in A
\]
es un homomorfismo de \(A\)-módulos.

Tenemos \(\Im f = Am\) y
\(\ann(a)=\ker f=\{a\in A: am=0\}\) es un ideal izquierda y se tiene
\[
  A/\ann_A(m)\cong Am
\]


\[
  a+\ann_A(m)\mapsto am
\]

Ejemplo: \(S=\Map(\N,K)\), el conjunto de las sucesiones (que forman
un \(K\)-espacio vectorial). Tomamos \(T:S\longrightarrow S\)
tal que \(T(s)(n)=s(n+1)\). Es lineal. Entonces
\(\subscriptbefore{K[x]}{S}\), donde \(xs=T(s)\).

Para cualquier \(f\in K[x]\), es decir \(f=\sum_i f_i x^i\), se tiene:
\[
  (fs)(n)=\sum_i f_i s(n+i)
\]

Imaginémosnos que \(s\) verifica que \(\ann_{K[x]}(s)\neq\langle0\rangle\).
Podemos tomar entonces un polinomio tal que \(fs = 0\) y que sea mónico.
Tenemos entonces que \(s(n+m)=-\sum_{i=0}^{m-1} f_i s(n+i)\)
para todo \(n\in\N\). Es decir, la sucesión es linealmente recursiva.

Caso particular, \(s(0)=s(1)=1\), tenemos que
\[
  s(n+2)=s(n)+s(n+1)
\]

\[
  x^2-x-1\in\ann_{\Q[x]}(s)
\]

Volviendo al caso general, tenemos que
\[
  K[x]/\ann_{K[x]}(s)\cong K[x]s
\]

Tenemos que \(\dim_{K}(K[x]s)<\infty\) si y solo si
\(\ann_{K[x]}(s)\neq\langle0\rangle\) si y solo si
\(s\) es una sucesión linealmente recursiva.

El generador \(p(x)\) de \(\ann_{K[x]}(s)\) se le llama el polinomio
mínimo de \(s\). El grado de dicho polinomio, coincide con
\(\dim_{k}(K[x]s)\) y se le llama complejidad lineal de \(s\).

\(s,t\) dos sucesiones linealmente recursivas.
\(K[x](s+t)\subseteq K[x]s+K[x]t\), luego la primera tiene dimensión finita.
Luego \(s+t\) es una sucesión linealmente recursiva, de complejidad menor
o igual a la suma de las complejidades lineales.
Puede argumentarse lo mismo para combinaciones lineales.

Las sucesiones linealmente recursivas forman un subespacio vectorial
del espacio de sucesiones. De hecho forman un submódulo. Sea
\(S^l\) el conjunto de las sucesiones linealmente recursivas, forma
un \(S^l\) es un \(K[x]\)-submódulo de \(S\), ya que es ivariante por la
acción de \(x\) (es \(T\)-invariante).

Otro ejemplo: \(T\) endomorfismo de \(\Cont^\infty(\R)\) tal que
\(T(\varphi)=\varphi'\). Tenemos que
\(\subscriptbefore{R[x]}{\Cont^\infty(\R)}\). Dada \(\varphi\),
tenemos que
\[
  \ann_{\R[x]}(\varphi)=\{f\in\R[x]: f(x)\varphi=0\}
  =\{f=\sum_i f_i\frac{d^i}{dt^i}: f\varphi=0\}
\]
\(\ann(\varphi)\neq\langle 0\rangle\) si \(\varphi\) satisface una ecuación
diferencial lineal homogénea con coeficientes constantes. Bla bla.

\(\R[x]/\ann_{\R[x]}(\varphi)\cong \R[x]\varphi\), donde \(\varphi\)
satisface bla bla.

Tenemos que \(\varphi''-\varphi'-\varphi=0\), cuya solución
\(\varphi(t)=e^{\phi t}\), donde \(\phi\) es la razón aúrea.


\subsubsection{Suma directa externa}

\begin{df}
Tomando el producto cartesiano de \(t\) módulos sobre el mismo anillo
y tomando la suma usual de tuplas y definiendo el siguiente producto:
\[
  a(m_1,\ldots, m_t)=(am_1,\ldots,am_t)
\]

Es un módulo que se llama suma directa externa de \(M_1,\ldots, M_t\)
con \(M^t\) si son todos iguales.

  Se denota \(M_1\oplus\cdots\oplus M_t\).
\end{df}


Ejercicio: Sea \(\subscriptbefore{A}{M}\), \(N_1,\ldots,
N_t\in\mathcal{L}(\subscriptbefore{A}{M})\). Se pide demostrar que existe
un homomorfismo \(f:N_1\oplus\cdots\oplus N_t\longrightarrow
N_1{+}\cdots{+}N_t\)
sobreyectivo de \(A\)-módulos tal que entre la suma directa
externa y la suma interna, tal que \(f\) es un isomorfismo si y solo si
la suma interna es directa. Podría ser interesante usar coordenadas.

\begin{df}[Base de un módulo libre]
  Consideramos \(A^n=A\oplus\cdots\oplus A\), donde la suma se repite
  \(n\) veces. Para cada \(i=1,\ldots, n\), tenemos que
  \(\{e_i: e_i=(0,\ldots, 0,1,0, \ldots, 0)\}\) forman un sistema de
  generadores de \(A^n\). Por tanto \(a=\sum_i a_i e_i\in A^n\)
  es una expresión única.
\end{df}

Dicha base puede no existir.

\begin{prop}
  Dado un módulo cualquiera \(\subscriptbefore{A}{M}\) y \(m_1, m_n\in M\),
  existe un único homomorfismo de módulos \(f:A^n\longrightarrow M\)
  tal que \(f(e_i)=m_i\).
\end{prop}

\begin{cor}
  Si \(M\) es finitamente generado con generadores \(\{m_i\}\),
  entonces \(M\cong A^n/L\) para \(L\) un sierto submódulo.
\end{cor}

\begin{proof}
  Unicidad: si existe una tal aplicación \(f\), entonces para
  cualquier \(a\in A^n\),
  \[
    f(a)=\sum_i a_i f(e_i)=\sum_i a_i m_i
  \]

  Veamos la existencia,
  Definiendo \(f(a)=\sum_i a_i m_i\) obtenemos un homomorfismo de módulos
  que cumple lo exigido en el enunciado.

  Si \(M=Am_1+\cdots+Am_n\) tenemos que \(L=\ker f\) cumple lo que se
  pide por el teorema de isomorfía para módulos.
\end{proof}


\section{Módulos Noetherianos}
\subsection{Álgebra homológica}

\begin{df}[Sucesiones exactas]
  Una suceión de homomorfismos de módulos \(f_i:M_i\longrightarrow
  M_{i+1}\) se dice exacta en
  \(M_{i+1}\) si \(\ker f_{i+1}=\Im f_i\).
\end{df}

Ejemplo: Dada una sucesión \(\{0\}\longrightarrow L \alpha
\longrightarrow M \beta\longrightarrow N\longrightarrow \{0\}\)
es exacta en \(L\) si y solo si \(\ker \alpha=\{0\}\), es decir,
\(\alpha\) es inyectiva, en \(N\) si y solo si \(\Im \beta = N\),
es decir, \(\beta\) sobreyectiva y en \(M\) si y solo si
\(\ker\beta=\Im\alpha\).

A \(\alpha\) se les llama monomorfismos de módulos y a
\(\beta\) epimorfismos de módulos.

A esta sucesión se le llama sucesión exacta corta.

Caso particular: Por ejemplo, si \(f:M\longrightarrow N\) es
un homorfismo de módulos, obtenemos:
\[
  0\longrightarrow\ker f\iota\longrightarrow M f\longrightarrow\Im f
  \longrightarrow 0
\]

\begin{prop}
  Sea \(0\longrightarrow L\overset{\psi}{\longrightarrow} M
  \overset{\varphi}{\longrightarrow}
  N\longrightarrow 0\) una sucesión exacta de \(A\)-módulos. Entonces:
  \begin{enumerate}
    \item Si \(M\) es finitamente generado, lo es también \(N\).
    \item Si \(L\) y \(N\) son finitamente generados, lo es también \(M\).
  \end{enumerate}
\end{prop}

\begin{proof}
  Veamos primero la primera afirmación. Sea \(\{m_i\}\) generadores de \(M\).
  Es claro que \(\{\varphi(m_i)\}\) generan \(N\).

  Para la segunda, \(\{n_i\}\) generadores de \(N\), y tomamos
  \(\{m_i\}\subseteq M\) tales que \(\varphi(m_i)= n_i\).

  Tomamos \(\{e_i\}\) generadores de \(L\). Tomamos \(m\in M\).
  \[
    \varphi(m)=\sum_{i=1}^s r_i n_i = \sum_{i=1} r_i\varphi(m_i)
    =\varphi\left(\sum r_i m_i\right)
  \]
  con lo que \(m-\varphi(\sum r_i m_i)\in\ker\varphi=\Im\psi\).
  Luego existen \(b_1,\ldots, b_t\) tales que
  \[
    m-\varphi\left(\sum r_i m_i\right)=
    \psi\left(\sum_j b_j e_j\right)
  \]
  y finalmente:
  \[
    m=\varphi\left(\sum r_i m_i\right)+\sum r_j \varphi(e_j)
  \]
  con lo que \(\{m_i\}\cup\{\psi(e_j)\}\).
\end{proof}

Ejemplo de que no se puede mejorar la proposición anterior:
Sea \(I\) un conjunto infinito, \(K\) un cuerpo.
\[
  K^I=\{{(\alpha_i)}_i\in I:\alpha_i\in K\}
\]
\(K^I\) es un anillo finitamente generado
por \((\ldots,1,1,1,\ldots)\). Definimos:
\[
  K^{(I)}=\{{(\alpha_i)}_i\in I:\alpha_i\in K \textrm{ y } \alpha_i=0
  \textrm{ salvo un número finito de } i\in I\}
\]

Tenemos que \(K^{(I)}\) es un ideal de \(K^I\), y por tanto ideal a
izquierda, pero no es finitamente generado como ideal a izquierda.

Es decir, \(M\) finitamente generado no implica que un submódulo suyo
sea finitamente generado.

\begin{df}[Módulos Noetherianos]
  Un módulo finitamente generado \(M\) se dice Noetheriano si todo
  submódulo de \(M\) es finitamente generado.
\end{df}

El ejemplo anterior no era un módulo Noetheriano.

\begin{prop}
  Equivalen:
  \begin{enumerate}
    \item \(M\) es noetheriano.
    \item Cualquier cadena ascendente \(L_1\subseteq L_2\subseteq\ldots
      \subseteq L_n\subseteq\ldots\) se estabiliza, es decir,
      a partir de un cierto \(m\) las inclusiones se vuelven igualdades.
    \item Cada subconjunto no vacío de \(\mathcal{L}(M)\) tiene un elemento
      maximal con respecto de la inclusión.
  \end{enumerate}
\end{prop}

\begin{proof}
  Veamos que la primera implica la segunda.
  Tomamos:
  \[
    L=\bigcup_{n\ge 1} L_n\in\mathcal{L}(M)
  \]
  es un submódulo porque están encajados. Por hipótesis, es finitamente
  generado. Si tomamos un conjunto finito de generadores \(F\)
  tenemos que \(F\subset L\) y como es finito, debe existir un \(m\)
  suficientemente grande tal que \(F\subseteq L_m\) y como
  genera a \(F\) se tiene que \(L\subseteq L_m\subseteq L\)
  con lo que \(L_n=L_m=L\) para todo \(n\ge m\).

  Veamos que la segunda implica la primera. Sea \(\Gamma\subseteq
  \mathcal{L}(M)\) no vacío. Si \(\Gamma\) no tiene elemento maximal
  y tomamos \(L_1\in\Gamma\), entonces existe \(L_2\in\Gamma\)
  tal que \(L_1\subsetneq L_2\).

  Reiterando el proceso, tenemos que \(L_1\subsetneq L_2\subsetneq
  \ldots\subsetneq L_n\subsetneq\ldots\) no se estabiliza.

  Veamos que la tercera afirmación implica la primera.
  Sea \(N\in\mathcal{L}(M)\).
  Tomamos el conjunto \(\Gamma\) el conjunto de todos los submódulos
  finitamente generados de \(N\). Tenemos que el módulo trivial
  es finitamente generado, luego \(\Gamma\) es no vacío.

  Sea \(L\) un elemento maximal de \(\Gamma\). Veamos que \(L=N\).

  En caso contrario, tomamos \(x\in N\) tal que \(x\notin L\). Resulta que
  \(L+Ax\) es un submódulo de \(N\) y es finitamente generado.
  \(L+Ax\in\Gamma\) y \(L\neq L+Ax\), con lo que \(L\) no sería maximal.
\end{proof}

\begin{nt}
  \(N\in\mathcal{L}(M)\), escribimos \(N\le M\).
\end{nt}

\begin{prop}[Sucesiones exactas cortas en módulos noetherianos]
  Sea \(0\longrightarrow L\overset{\varphi}{\longrightarrow}
  M\overset{\psi}{\longrightarrow} N
  \longrightarrow 0\).

  Entonces \(M\) es noetheriano si y solo si \(L\) y \(N\) son
  noetherianos.
\end{prop}

\begin{proof}
  Supongamos \(M\) noetheriano.

  \(L\cong \Im\psi\le M\) y entonces \(L\) es noetheriano trivialmente.

  Tomamos \(N_1\subseteq N_2\subseteq\ldots\subseteq N_n\subseteq\ldots\)
  una cadena ascendente en \(\mathcal{L}(N)\).

  Tenemos \(\varphi^{-1}(N_1)\subseteq \varphi^{-1}(N_2)
  \subseteq \varphi^{-1}(N_n)\subseteq\ldots\)
  cadena en \(\mathcal{L}(M)\). Existe un \(m\) a partir del cual
  se estabiliza. Entonces, para todo \(n\ge n\):
  \[
    N_n=\varphi(\varphi^{-1}(N_n))=\varphi(\varphi^{-1}(N_m))=N_m
  \]
  con lo cual \(N\) es noetheriano.


  Supongamos ahora que \(N\) y \(L\) son noeherianos.
  Tomamos una cadena ascendente \(M_n\) de submódulos de \(M\).

  Por otro lado, \(M_n\cap\Im\psi\) es una cadena de submódulos de
  \(M\), que se estabiliza por ser noetheriano \(\Im\psi\cong L\).

  Tenemos \(\varphi(M_n)\) es una cadena de submódulos de \(N\),
  que también se estabiliza.

  Tomemos el menor natural tal que ambas cadenas se hayan estabilizado.
  Sea \(n\) mayor, \(x\in M_n\), \(\varphi(x)\in\varphi(M_n)
  =\varphi(M_m)\), debe existir \(y\in M_m\). Luego \(x-y\in\ker\varphi
  =\Im\psi\), con lo que \(x-y\in M_n\cap\Im\psi=M_m\cap\Im\psi\subseteq M_m\)
  y \(x\in M_m\) ya que \(y\in M_m\).

  Por tanto \(M\) es noetheriano.
\end{proof}

\begin{cor}
  Dados dos módulos \(M_1\) y \(M_2\).
  Entonces:
  \[
    M_1\oplus M_2
  \]
  es noetheriano si y solo si \(M_1\) y \(M_2\) lo son.
\end{cor}

\begin{proof}
  Sea la sucesión exacta corta
  \[
    0\longrightarrow M_1\longrightarrow M_1\oplus M_2\longrightarrow M_2
    \longrightarrow 0
  \]
  donde la primera aplicación es \(m_1\mapsto(m_1,0)\)
  y \((m_1,m_2)\mapsto m_2\) y el núcleo de la segunda es la imagen de
  la primera. Trivialmente se sigue el corolario.
\end{proof}

\begin{teo}
  Sea \(A\) un anillo. Cada módulo sobre \(A\) finitamente generado
  es noetheriano si y solo si \(\subscriptbefore{A}{A}\) es noetheriano.
\end{teo}

\begin{proof}
  Una de las implicaciones es obvia.

  Veamos que si el módulo regular es noetheriano, veamos que cualquier otro
  lo es.

  Sea \(M\) finitamente generado, existe un homomorfismo sobreyectivo \(\phi\)
  tal que \(A^n\longrightarrow M\).

  Usando inductivamente el corolario, tenemos que \(A^n\) es noetheriano.
  La proposición nos dice que \(M\) es noetheriano, aplicandolo
  a la sucesión
  \[
    0\longrightarrow\ker\phi\longrightarrow A^n\longrightarrow
    M\longrightarrow 0
  \]
\end{proof}

\begin{df}[Anillo noetheriano]
  \(A\) se dice noetheriano a izquierda si el módulo regular es
  noetheriano. Si \(A\) es conmutativo diremos simplemente noetheriano.
\end{df}

\begin{cor}
  Si \(A\) es noetheriano, equivalen para cualquier sucesión exacta corta:
  \begin{enumerate}
    \item \(M\) es finitamente generado.
    \item \(L\) y \(N\) son finitamente generados.
  \end{enumerate}
\end{cor}

\begin{cor}
  Todo dominio de ideales principales es noetheriano.
\end{cor}

\subsection{Módulo Artiniano}

\begin{df}[Módulo artinano]
  Para un \(\subscriptbefore{A}{M}\), son equivalentes:
  \begin{enumerate}
    \item Cada cadena descendente
      \(L_1\supseteq L_2\supseteq\ldots\supseteq L_n\supseteq\ldots\)
      de submódulos de \(M\)
      se estabiliza, esto es, a partir de cierto natural \(m\)
      se tiene \(L_n=L_m\) para todo \(n\ge m\).
    \item Cada subconjunto de \(\mathcal{L}(M)\) tiene un elemento
      minimal.
  \end{enumerate}
  A un tal módulo lo llamaremos artiniano.
\end{df}

Ejercicio: Sea \(A\) un dominio de integridad conmutativo. Si el
módulo regular es artiniano, entonces \(A\) es un cuerpo.

En particular \(\Z\) no es artiniano, aunque por ser un DIP, sí que
es noetheriano.

Ejercicio: \(K\) un cuerpo de característica 0. Tomo \(K[x]\) anillo
de polinomios. Veo \(K[x]\) como \(K\)-espacio vectorial.
Tomamos \(T\) la aplicación lineal \(T(f):=f'\), donde \(f'\) es el
polinomio derivado. Esto nos da una estructura de \(K[x]\)-módulo
sobre \(K[x]\) que no es la del módulo regular. Se pide demostrar
que ese módulo es artiniano y no finitamente generado.

En consecuencia, la estructura que hemos definido no es la misma
que la del módulo regular.

\begin{prop}
  Sea \[0\longrightarrow L\longrightarrow M\longrightarrow N\longrightarrow
  0\]

  Entonces \(M\) es artiniano si y solo si \(L\) y \(N\) son artinianos.
\end{prop}

Ejercicio: sea \(p\) un número primo. Definimos:
\[
  C_{p^\infty}=\{z\in\C:z^{p^n}=1\textrm{ para algún } n\ge1\}
\]
Se pide comprobar que es un subgrupo \(\S=\{z\in\C:\md{z}=1\}\) y demostrar
que visto como \(\Z\)-módulo es artiniano pero no es finitamente generado.


\subsection{Módulos de longitud finita}

\begin{df}[Serie de composición]
  Sea \(M\) un módulo. Una serie de composición de \(M\)
  es una cadena de submódulos
  \[
    M=M_n\supsetneq M_{n-1}\supsetneq\ldots\supsetneq M_1
    \supsetneq M_0=\{0\}
  \]
  tal que si \(M_i\supseteq N\supseteq M_{i-1}\) para \(N\) submódulo,
  entonces \(N=M_i\) o \(N=M_{i-1}\). Es decir, cada submódulo es maximal
  en el anterior.

  A \(n\) le llamamos la longitud de la serie.
\end{df}

Ejemplo: serie de composición de \(\Z_{12}\). Tiene como subgrupos
a \(\Z_m\) con \(m\) divisor de 12.
\[
  M_3=\Z_{12}
\]
tiene como subgrupo maximal (argumentando por Lagrange):
\[
  M_2=\langle 2 \rangle
\]
que a su vez tiene como subgrupo maximal
\[
  M_1=\langle 4\rangle
\]
y ya solo tiene
\[
  M_0=\{0\}
\]

\begin{df}[Módulo simple]
  \(M\) se dice simple si \(M\supset\{0\}\) es una serie de composición.
  Es decir, si no tiene submódulos propios y no es el módulo 0.
\end{df}

\begin{prop}
  La condición de que cada submódulo sea maximal en el anterior es equivalente
  a que los factores \(M_i/M_{i-1}\) sean simples.
\end{prop}

\begin{teo}
  Toda serie de composición del mismo módulo tiene la misma longitud
  y los mismos factores salvo isomorfismo y reordenación.
\end{teo}

\(\Z_{12}\) tiene como factores \(\Z_2\), \(\Z_2\) y \(\Z_3\).

\begin{prop}
  Un módulo no nulo admite una serie de composición si y solo si es
  noetheriano y artiniano.
\end{prop}
\begin{proof}
  Sea \(M_i\) una serie de composición. Inducción sobre \(n\).
  Si \(n=1\), tenemos que \(M\) es simple y en particular noetheriano
  y artiniano.

  Si \(n>1\), entonces \(M_{n-1}\) admite una serie de composición de
  longitud \(n-1\), luego es noetheriano y artiniano. Tomamos
  la sucesión exacta corta \[0\longrightarrow M_{n-1}
  \longrightarrow M_n\longrightarrow M_n/M_{n-1}\longrightarrow 0\]
  El primer elemento es noetheriano y artiniano, el último es simple (luego
  noetheriano y artiniano), con lo que \(M_n\) es noetheriano y artiniano.

  Para el recíproco, como \(M\) es artiniano, contiene un submódulo
  simple \(M_1\). Entonces hay un \(M_2\supsetneq M_1\) donde
  \(M_2/M_1\) es simple. Reiterando el proceso, tenemos
  \(0\subsetneq M_1\subsetneq M_2\subsetneq \ldots\) y como es
  noetheriano, habrá un \(M_n\) que termine la cadena.
\end{proof}

\begin{cor}
  Dada una sucesión exacta corta, \(0\longrightarrow L
  \longrightarrow M\longrightarrow N\longrightarrow 0\),
  \(L\) y \(N\) admite serie de composición si y solo si
  \(M\) admite serie de composición.
\end{cor}

\begin{cor}
  \(M_1\), \(M_2\) admiten series de composición si y solo si
  \(M_1\oplus M_2\) admite serie de composición.
\end{cor}


\begin{teo}[Jordan-Hölder]
  Supongan que \(M\) admite series de composición:
  \[
    \{0\}=M_0\subsetneq M_1\subsetneq M_2\subsetneq\ldots\subsetneq M_n=M
  \]
  \[
    \{0\}=N_0\subsetneq N_1\subsetneq N_2\subsetneq\ldots\subsetneq N_m=M
  \]
  Entonces \(n=m\) y existe una permutación \(\sigma\) tal que
  \[
    M_i/M_{i-1}\cong N_{\sigma(i)}/N_{\sigma(i)-1}
  \]
\end{teo}
\begin{proof}
  Si \(n=1\), entonces \(M\) es simple y \(m=1\) y el el único factor posible
  es el \(M/\{0\}=M\).

  Si \(n>1\), como \(M\) no es simple, \(m>1\).

  Vamos a observar un caso particular. Supongamos que \(N_{m-1}
  =M_{n-1}\). Por hipótesis de inducción aplicado a \(N_{m-1}\),
  tenemos que  \(n-1=m-1\), luego \(n=m\) y se da el enunciado
  (tomando la permutación \(\sigma\) para los \(n-1\) primeros elementos
  y extendiendola a una permutación de \(n\) elementos \(\sigma'\) tal
  que \(\sigma'(n):=n\), \(\sigma'(k):=\sigma(k)\)).

  Vamos ahora al caso general. Como hemos visto
  en el caso particular anterior, podemos suponer
  \(M_{n-1}\neq N_{m-1}\), por lo que
  \(M_{n-1}+M_{m-1}=M\) (ya que \(M_{n-1}\subsetneq M_{n-1}+
  N_{m-1}\subseteq M\) y \(M_{n-1}\) es maximal).

  Tomamos \(N_{m-1}\cap M_{n-1}\) que admite una serie de composición:
  \[
    \{0\}=L_0\subsetneq L_1\subsetneq\ldots\subsetneq L_k
    =N_{m-1}\cap M_{n-1}
  \]
  y tenemos que, por el teorema de isomorfía:
  \[
    N_m/N_{m-1}=
    M/N_{m-1}=
    (M_{n-1}+N_{m-1})/N_{m-1}\cong M_{n-1}/(M_{n-1}\cap N_{m-1})
  \]
  que al ser un factor es simple.

  Aplicando la inducción, \(n-1=k+1\) y existe una permutación
  \(\tau\) de \(n-1\) elementos tal que
  \[
    L_i/L_{i-1}\cong M_{\tau(i)}/M_{\tau(i)-1}
  \]
  donde \(i=1,\ldots, n-2\)
  y
  \[
    M_{n-1}/L_{n-2}=M_{n-1}/(M_{n-1}\cap N_{m-1})\cong
    M_{\tau(n-1)}/M_{\tau(n-1)-1}
  \]

  Tenemos que, por el teorema de isomorfía:
  \[
    M_n/M_{n-1}=
    M/M_{n-1}=
    (N_{m-1}+M_{n-1})/M_{n-1}\cong N_{m-1}/(N_{m-1}\cap M_{n-1})
  \]
  que al ser un factor es simple.

  Aplicando la inducción, \(m-1=k+1\) y existe una permutación
  \(\rho\) de \(m-1\) elementos tal que
  \[
    L_i/L_{i-1}\cong N_{\rho(i)}/N_{\rho(i)-1}
  \]
  donde \(i=1,\ldots, n-2\)
  y
  \[
    N_{n-1}/L_{n-2}=N_{n-1}/(M_{n-1}\cap N_{m-1})\cong
    N_{\rho(n-1)}/N_{\rho(n-1)-1}
  \]

  Tenemos ya que \(n=k+2=m\), y si definimos \(\sigma\) la permutación
  de \(n\) elementos:
  \[
    \sigma(i)=\left\{
      \begin{matrix}
        \rho\circ\tau^{-1}(i),
        &i\in\{1,\ldots n-1\}, &\tau^{-1}(i)\in\{1,\ldots, n-2\}\\
        n,
        &i\in\{1,\ldots n-1\}, &\tau^{-1}(i)=n-1\\
        \rho(n-1),
        &i=n &
      \end{matrix}
      \right.
  \]
\end{proof}

\begin{df}[Módulo de longitud finita]
  Un módulo se dice de longitud finita si tiene una serie de composición
  finita o es \(\{0\}\). La longitud \(\ell(M)\) es la de cualquiera
  de sus series de composición, o cero si \(M=\{0\}\).
\end{df}

Ejercicio: sea \(M\) un módulo de longitud finita. Se pide demostrar
que si \(0\longrightarrow L\longrightarrow M\longrightarrow N\longrightarrow
0\) es una sucesión exacta corta, entonces:
\[
  \ell(M)=\ell(N)+\ell(M)
\]
Si \(U,V\in\mathcal{L}(M)\), entonces:
\[
  \ell(U+V)=\ell(U)+\ell(V)-\ell(U\cap V)
\]

Ejemplo: si \(V\) es un \(K\)-espacio vectorial, \(\ell(V)=\dim(V)\).

Ejemplo: \(\ell(\Z_{12})=3\), ya que calculamos antes una serie de
composición.

Otro ejemplo: \(\ell(\Z_p)=1\) si \(p\) es primo.

Ejercicio: \(\ell(\Z_n)\) es la suma de los exponentes de su descomposición
en primos.

Ejemplo: si \(n=\prod {p_i}^{e_i}\) entonces\(\Z_n\cong
\Z_{{p_t}^{e_t}}\oplus\cdots
\oplus\Z_{{p_t}^{e_t}}\)

Sea \(\subscriptbefore{A}{M}\) un módulo, \(\mathcal{L}(M)\)
es el conjunto de todos los submódulos de \(M\).

Dado \(\Gamma\subseteq\mathcal{L}(L)\) no vacío,
tenemos \(\bigcap_{N\in\Gamma}
N\in\mathcal{L}(M)\) (no tiene por qué ocurrir que estén en \(\Gamma\),
\(\bigcap_{n\ge 1} n\Z=\{0\}\notin m\Z\) para ningún \(m\ge 1\)).


\begin{df}[Zócalo]
  El zócalo de \(M\) es el menor submódulo de \(M\) que contiene a todos
  los submódulos simples de \(M\).

  Si \(M\) no tiene ningún submodulo simple, definimos el zócalo como
  \(\{0\}\).

  En ambos casos usaremos la notación \(\Soc(M)\).
\end{df}

Ejemplo: si \(V\) es un \(K\)-espacio vectorial, \(\Soc(V)=V\).

Ejemplo: \(\Soc(\Z)=\{0\}\), puesto que cada \(n\Z\) contiene
un \(2n\Z\), luego no es simple.

De hecho, si \(A\) es un dominio de
integridad que no es un cuerpo, \(\Soc(A)=\{0\}\). Tienes que sus submódulos
son ideales. Para \(x\in I\), el ideal generado por \(x^2\) está dentro de
\(I\), luego \(I\) no es simple.

\begin{prop}
  Sea \(M\) de longitud finita. Existen submódulos \(S_i\) simples de \(M\)
  tales que
  \[
    \Soc(M)=S_1\dot{+}\cdots\dot{+} S_n
  \]
  Además si \(T_i\) son simples tales que
  \(
    \Soc(M)=T_1\dot{+}\cdots\dot{+} T_m
  \), entonces \(n=m\) y tras reordenación, \(S_i\cong T_i\).
\end{prop}

\begin{proof}
  Si \(\Gamma\) es el conjunto de todos los submódulos de la forma
  \(
    S_1\dot{+}\cdots\dot{+} S_n
  \)

  Si \(M\neq\{0\}\), entonces \(\Gamma\neq\emptyset\), ya que \(M\)
  contiene algún submódulo simple.

  Como \(M\) es Notheriano, existe un
  \(
    S_1\dot{+}\cdots\dot{+} S_n
  \) maximal.

  \(
    S_1\dot{+}\cdots\dot{+} S_n\subseteq\Soc(M)
  \). Sea \(S\in\mathcal{L}(M)\) simple.
  \[
    S\cap(
    S_1\dot{+}\cdots\dot{+} S_n
    )
  \]
  puesto que \(S\) es simple y la intersección es submódulo, se tiene
  que dicha intersección o es \(\{0\}\) o es \(S\).

  Consideramos
  \[
    S\cap(
    S_1\dot{+}\cdots\dot{+} S_n
    ) = \{0\}
  \]
  luego
  \[
    S\dot{+}S_1\dot{+}\cdots\dot{+} S_n\in\Gamma
  \]
  con lo que no sería maximal.

  Luego se tiene:
  \[
    S\subseteq
    S_1\dot{+}\cdots\dot{+} S_n\in\Gamma
  \]
  luego, como \(S\) era un modulo simple arbitrario, tenemos que
  \(\Soc(M)=
    S_1\dot{+}\cdots\dot{+} S_n
  \).

  Resulta que
  \[
    \{0\}\subsetneq S_1\subsetneq S_1\dot{+} S_2\subsetneq\ldots
    \subsetneq
    S_1\dot{+}\cdots\dot{+} S_n=\Soc(M)
  \]
  es una serie de composición, ya que:
  \[
    (S_1\dot{+}\cdots\dot{+} S_i)/
    (S_1\dot{+}\cdots\dot{+} S_{i-1})\cong
    S_i
  \]
  Aplicando Jordan-Hölder se obtiene el resultado.
\end{proof}

\begin{df}[Módulo semisimple]
  Sea \(M\) de longitud finita. Decimos que \(M\) es semisimple si es
  \(\Soc(M)=M\).
\end{df}

Ejercicio: Sea \(A\) un DIP que no sea un cuerpo,
\(I\) ideal de \(A\). Se pide demostrar
que \(A/I\) es de longitud finita si y solo si \(I\neq\langle 0\rangle\).

¿Se puede deducir cuál es la longitud de \(A/I\) de un generador de \(I\)?

\subsubsection{Módulos de longitud finita sobre un DIP}

Sea de ahora en adelante \(A\) un dominio de ideales principales que no
sea un cuerpo.

\begin{lema}
  \(\subscriptbefore{A}{M}\) es de longitud finita si y solo si
  \(\subscriptbefore{A}{M}\) finitamente generado y acotado.
\end{lema}
\begin{proof}
  \(M\) distinto del 0, porque si no es trivial.

  \(M\) de longitud finita, por tanto noetheriano, por tanto finitamente
  generado: \(M=Am_1+\cdots+Am_n\), con \(m_i\in M\).
  \[
    \langle\mu\rangle\Ann_A(M)=\bigcap_{i=1}^n\ann_A(m_i)
  \]
  porque el anillo \(A\) es conmutativo, donde ademas cada anulador
  de cada elemento es un ideal (a izquierdas en un conmutativo, luego ideal).

  Sea \(\langle f_i\rangle\ann_A(m_i)\), entonces
  \[
    \langle\mu\rangle=\bigcap_{i=1}^n\langle f_i\rangle
  \]
  donde \(\mu=\mcm\{f_i:1\le i\le n\}\).

  Veamos que \(f_i\neq 0\) para cada \(i\).
  \[
    M\subseteq Am_i\cong A/\langle f_i\rangle
  \]
  luego \(\ell(Am_i)<\infty\), como \(A\) no es un cuerpo y por
  tanto \(M\) no es artiniano, entonces
  \(\langle f_i\rangle\neq0\).

  Luego \(\langle\mu\rangle\neq 0\) y por tanto \(M\) es acotado.

  Veamos el recíproco: \(M\) acotado y finitamente generado.
  \[
    M=Am_1+\cdots+Am_n
  \]

  Vemos que cada \(Am_i\) es de longitud finita (\(\mu\neq 0\) por ser
  acotado, luego cada \(\langle f_i\rangle\neq 0\)).
  Tenmos que \(Am_i\cong A/\langle f_i\rangle\) es de longitud finita.

  Existe un epimorfismo entre \(Am_1\oplus\cdots\oplus Am_n\) (que es
  de longitud finita) y \(Am_1\oplus\cdots\oplus Am_n\), con lo que
  el segundo tiene longitud finita.
\end{proof}

\(\ell_A(M)<\infty\), entonces es acotado, o sea
\(\langle\mu\rangle=\Ann_A(M)=\langle 0\rangle\). Entonces
\[
  M=M_1\dot{+}\cdots\dot{+} M_t
\]
donde \(M_i\) es la componente \(p_i\) primaria que viene de
\(\mu={p_1}^{e_1}\cdots{p_t}^{e_t}\)
(\(M_i=\{m\in M:m\cdot{p_i}^{e_i}=0\}\)).
Además \(M_i\) es finitamente generado.
¿Se puede descomponer como suma directa de submódulos indescomponibles?


\[
  M=M_1\dot{+}\cdots\dot{+} M_t
\]
donde
\[
  M_i=\{q_i m: m\in M\}=\{m\in M: p_i^{e_i} m=0\}=\{m\in M:
  a_i q_i m = m\}
\]
con \(q_i=\frac{\mu}{p_i^{e_i}}\) y \(\sum_i a_i q_i=1\)
y \(\langle \mu\rangle=\Ann_A(M)\). Se tiene que
\(\Ann_A(M_i)=\langle p_i^{e_i}\rangle\).

\begin{df}[Módulo p-primario]
  \(\subscriptbefore{A}{M}\) se dice \(p\)-primario si
  \(\Ann_A(M)=\langle p^e\rangle\), \(p\) un irreducible.
\end{df}

Vamos a estudiar la estructura de módulos primarios de longitud finita.

\begin{obs}
  \(\subscriptbefore{A}{M}\) \(p\)-primario, \(\ell(M)<\infty\).
  \[
    \Ann_A(M)=\langle p^t\rangle
  \]

  Si \(0\neq m\in M\), \(\ann_A(m)\supseteq \Ann_A(M)=\langle p^t
  \rangle\), tenemso que \(\ann_A(m)=\langle p^r\rangle \)
  con \(r\le t\).

  Si \(M=Am_1+\cdots+Am_m\), entonces \(\langle p^t\rangle
  =\ann_A(m_1)\cap\ldots\cap\ann_A(m_m)\). Luego
  \(\langle p^t\rangle =\ann_A(m_i)\) para algún \(i\).
\end{obs}

\begin{cor}
  Existe un \(x\i M\), \(\Ann_A(M)=\ann_A(x)\).
\end{cor}

\begin{lema}
  \(\ell(M)<\infty\), \(M\) \(p\)-primario. Para \(0\neq m\in M\),
  entonces:
  \[
    Am\textrm{ es simple} \iff \ann_A(m)=\langle p\rangle
  \]
  y como consecuencia
  \[
    \Soc(M)=\{m\in M: pm=0\}
  \]
\end{lema}
\begin{proof}
  Dado \(m\), tenemos \(Am\cong A/\ann_a(m)\). Si \(Am\) es simple,
  entonces \(\ann_A(m)\) es ideal maximal (generado por irreducible o ideal
  primo) y
  \(\ann_A(m)\supseteq\Ann_A(M)=\langle p^t\rangle\).
  Entonces \(\ann_A(m)=\langle p\rangle\).

  Recíprocamente, si \(\ann_A(m)=\langle p\rangle\) entonces
  \(Am\cong A/\langle p\rangle\) es simple.

  \(\Soc(M)=S_1\dot{+}\cdots\dot{+}S_n\) con \(S_i\) simple.
  Sea \(m\) en el zócalo, \(\ann_A(m)
  \supseteq\Ann_A(S_1\dot{+}\cdots\dot{+}S_n)=
  \bigcap_{k=1}^{n} \Ann_A(S_k)\). Tomamos \(s_i\) tal que
  \(\Ann_A(S_i)=\ann_A(s_i)\), tenemos que \(S_i=As_i\), luego
  \(As_i\cong A/\ann_A(s_i)\) y es simple, luego
  \(\ann_A(s_i)=\langle p\rangle\),
  tenemos que \(\ann_A(m)\supseteq\langle p\rangle\) y finalmente
  \(pm=0\).

  Tomamos ahora \(m\in M\) tal que \(pm=0\). \(\langle p\rangle
  \subseteq\ann_A(m)\) pero es maximal, luego se da la igualdad.
  \[
    Am\cong A/\ann_A(m)= A/\langle p\rangle
  \]
  luego es simple, y \(Am\subseteq\Soc(M)\) y en particular
  \(m\in\Soc(M)\).

\end{proof}

\begin{prop}
  Suponemos que tenemos \(M\) \(p\)-primario y de longitud finita. Sea
  \(x\in M\) tal que \(\Ann_A(M)=\ann_A(x)\). Entonces \(Ax\) es un
  sumando directo interno de \(M\).
\end{prop}
\begin{proof}
  Por inducción sobre la longitud \(\ell(M)<\infty\).

  Si la longitud es 1, \(M\) es simple, entonces \(M=Ax\).

  Si \(\ell(M)>1\) y \(Ax=M\), no hay nada que demostrar.

  Veamos que pasa si \(Ax\neq M\). Veamos que existe un \(y \in M\)
  tal que \(y\neq Ax\) y \(\ann_A(y)=\langle p\rangle\).
  \(
    \ell(M/Ax)<\infty
  \), debe contener algún simple \(S\subseteq M/Ax\). Tomamos \(s\in S\)
  tal que \(S=As\).
  \[
    \langle p^t\rangle =\Ann_A(M)\subseteq\Ann_A(M/Ax)\subseteq\Ann_A(S)
    =\ann_A(s)
  \]
  Y por tanto \(\ann_A(s)=\langle p\rangle\).

  Tomamos \(z\in M\) tal que \(s=z+Ax\), es decir, \(p z\in Ax\). Es decir,
  \(p z=ax\) para cierto \(a\in A\). Afirmamos que \(p|a\) (no es obvio
  porque es un módulo).

  Supongamos que no es así. Por Bezout, \(1=ua+vp\) para \(u,v\in A\)
  adecuados. En dicho caso, \(x=uax+vpx=upz+vpx=p(uz+vx)\).
  \[
    \ann_A(uz+vx)=\langle p^{t'}\rangle
  \]
  para \(t'\le t\). Se deduce que \(p^{t'-1} x=0\). \(p^{t-1}x=0\), y
  entonces como el anulador de \(x\) es el de \(M\) y está generado
  por \(p^t\), no puede anularlo \(p^{t'-1}\) ya que
  \(t'-1\le t-1<t\).

  Cuenta alternativa: \(p^{t-1}ax=p^t z=0\) entonces \(p^{t-1} a
  \in\ann_A(x)=\langle p^t\rangle\), tenemos que \(a=pa'\)

  Hemos obtenido un elemento \(s=z+Ax\in M/Ax\) y que \(pz=ax\) y hemos
  visto que \(p|a\). Así tenemos que \(pz=pa'x\) y entonces
  \(p(z-a'x)=0\). Llamo \(y=z-a'x\neq 0\) y \(py=0\) con lo que
  \(\ann_A(y)=\langle p\rangle\).

  Tenemos que \(Ay\) es simple y \(y\notin Ax\) asi que \(Ay\cap Ax=
  \{0\}\).
  \[
    Ax\cong Ax/(Ay\cap Ax)\cong (Ax+Ay)/Ay\cong A(x+Ay)\subseteq M/Ay
  \]

  \[
    \langle p^t\rangle=\ann_A(x)=\ann_A(A(x+Ay))\supseteq
    \Ann_A(M/Ay)\supseteq\Ann_A(M)=\langle p^t\rangle
  \]
  con lo cual todas las inclusiones son igualdades.

  Tenemos que \(\Ann_A(M/Ay)=\langle p^t\rangle=\ann_A(x+Ay)\), que
  están en las mismas condiciones de la hipótesis pero
  con \(\ell(M/Ay)<\ell(M)\). Aplicando la hipótesis de inducción,
  tenemos que \(M/Ay=(Ax+Ay)/Ay \dot{+} N/Ay\) para cierto
  \(N\in\mathcal{L}(M)\) tal que \(N\supseteq Ay\). De aquí se deduce
  que \(M=Ax+Ay+N=Ax+N\). Tomamos \(Ax\cap N\subseteq(Ax+Ay)\cap N=Ay\).
  Entonces \(Ax\cap N = Ax\cap N\cap Ay= Ax\cap Ay=\{0\}\).

\end{proof}

\begin{teo}
  Sea \(\subscriptbefore{A}{M}\) \(p\)-primario de longitud finita. Existen
  \(x_1,\ldots, x_n\in M\setminus\{0\}\) tales
  que \(M=Ax_1\dot{+}\cdots\dot{+} Ax_n\) y
  \[
    \Ann_A(M)=\ann_A(x_1)\supseteq\ann_A(x_2)\supseteq\ldots
    \supseteq\ann_A(x_n)
  \]
  Además, si \(y_1,\ldots, y_n\in M\) no nulos son tales que
  \(
    M=Ay_1\dot{+}\ldots\dot{+} Ay_n
  \)
  y
  \(
    \Ann_A(M)=\ann_A(y_1)\supseteq\ann_A(y_2)\supseteq\ldots
    \supseteq\ann_A(y_m)
  \), entonces \(n=m\) y \(\ann_A(x_i)=\ann_A(y_i)\).
\end{teo}
\begin{proof}
  Tomo \(x_1\in M\) tal que \(\Ann_A(M)=\ann_A(x)\), por la proposición,
  \(M=Ax_1\dot{+} N\) para cierto submódulo \(N\) de \(M\).
  Es claro que \(\Ann_A(N)\supseteq\Ann_A(M)=\langle p^t\rangle\),
  con lo que \(\Ann_A(N)=\langle p^{t'}\rangle\) con \(t'\le t\) y
  \(\ell(N)<\ell(M)\).

  Por inducción sobre \(\ell(M)\), tenemos \(x_1,x_2,\ldots, x_n\in N\)
  y \(N=Ax_2\dot{+}\cdots\dot{+} Ax_n\).
  De esto se deduce
  \[
    M=Ax_1\dot{+}\cdots\dot{+} Ax_n
  \]
  y \(\ann_A(x_1)=\Ann_A(M)\subseteq\ann_A(x_2)\subseteq\ldots\subseteq
  \ann_A(x_n)\).

  Veamos la unicidad. Hacemos inducción sobre \(\ell(M)\).

  Si \(\ell(M)=1\), tenemso que es simple y \(M=Ax=Ay\) y \(n=1=m\).

  Si \(\ell(M)>1\), tenemos que \(M\) no es simple. Consideramos
  \(M/pM\) donde \(pM:=\{pm:m\in M\}\) que es un submódulo por ser
  \(A\) conmutativo. \(\Ann_A(pM)=\langle p\rangle\).
  \[
    \Soc(M/pM)=M/pM
  \]
  luego \(M/pM\) es semisimple.

  Tengo un homomorfismo de módulos \(M\longrightarrow
  Ax_1/Apx_1\oplus\cdots
  Ax_n/Apx_1
  \) tal que \(\sum A-ix_i\mapsto (a_1 x_1+Apx_1,\ldots,
  a_n x+Apx_n)\).

  Se puede demotrar que dicha aplicación es sobreyectivo y su núcleo es
  \(pM\).
  \[
    M/pM\cong
    Ax_1/Apx_1\oplus\cdots
    Ax_n/Apx_1
  \]
  \(n = \ell(M/pM)\). Argumentando de forma análoga para \(y\);
  obtenemos \(n= \ell(M/pM)=m\).

  Si \(pM=\{0\}\), tenemos que todos los anuladores son iguales:
  \(\ann_A(x_i)=\langle p\rangle=\ann_A(y_i)\).

  Supongamos que \(pM\neq\{0\}\).
  \[
    pM=Apx_1\dot{+}\cdots\dot{+} Apx_r
  \]
  para cierto \(r\le n\).

  Así, \(\ann_A(x_i)=\langle p\rangle\) si solo si \(i>r\).
  y también \(\ann_A(y_i)=\langle p\rangle\) si solo si \(i>r\).
  Para cualquier \(i\le r\), tenemos que \(\ann_A(px_i)=\langle
  p^{t_i-1}\rangle\) si \(\ann_A(x_i)=\langle p^{t_i}\rangle\).

  \[
   \ann_A(px_1)\supseteq\ann_A(px_2)\supseteq\ldots
    \supseteq\ann_A(px_r)
  \]
  \[
   \ann_A(py_1)\supseteq\ann_A(py_2)\supseteq\ldots
    \supseteq\ann_A(py_s)
  \]
  donde \(\ann_A(y_i)=\langle p^{s_i}\rangle\) si y solo si \(i>s\).
  Pero \(\ell(pM)<\ell(M)\), por inducción \(s=r\) y que \(s_i-1=
  r_i-1\) y como sabemos que si \(i>r=s\) tenemos que
  \(\ann_A(x_i)=\ann(y_i)=\langle p\rangle\).

\end{proof}

\begin{obs}
  Si \(A=\Z\), \(M\) grupo abeliano, \(x\in M\),
  \(\ann_\Z(x)=n\Z\), \(n\) recibe el nombre de el orden.
\end{obs}

\begin{obs}
  Si \(A=K[x]\), \(T:V\longrightarrow V\), \(n=\dim_K V<\infty\),
  \(v\in V\), \(\ann_{K[x]}(v)=\langle f(x)\rangle\).
  Tenemos que \(f\) tiene grado \(n\). \(\{v, Tv,\ldots, T^{n-1}v\}\)
  es una base de \(V\).
\end{obs}

Ejemplo: \(\mathcal{U}(\Z_8)=\{1,3,5,7\}\).
Viendo los ordenes de los elementos:
\[\mathcal{U}(\Z_8)=\langle 3\rangle\dot{+}\langle 5\rangle\]
donde \(\langle \cdot\rangle\) es la generación como subgrupo.

Ejemplo: Suponemos un espacio vectorial \(V\) de dimensión 3 y un
endomorfismo \(T\) cuyo polinomio mínimo es de la forma
\({(x-\lambda)}^2\) con \(\lambda\in K\).
Sabemos que existen dos vectores \(v_1,v_2\) tales que
\[
  V=K[x]v_1\dot{+} K[x]v_2
\]
con \(\ann_{K[x]} v=\langle {(x-\lambda)}^2\rangle
\subsetneq\langle {x-\lambda}\rangle=\ann_{K[x]} v_2\).

\begin{cor}
  Si \(\subscriptbefore{A}{M}\) es un módulo \(p\)-primario, entonces
  \[
    M\cong C_1\oplus\cdots\oplus C_n
  \]
  con \(C_i\) cíclico.

  Si \(M\cong D_1\oplus\cdots\oplus D_m\), con \(D_i\) cíclico, entonces
  \(n=m\) y tras reordenación, \(D_i\cong C_i\) para todo \(i\).
\end{cor}
\begin{proof}
  De \(M\cong C_1\oplus\cdots\oplus C_n\), se puede exigir que
  \(x_1,\ldots,x_n\in M\) tales que
  \[
    M=Ax_1\dot{+}\cdots\dot{+}Ax_n
  \]
  con \(\ann_A(x_1)\subseteq\ann_A(x_2)\subseteq\ldots\subseteq\ann_A(x_n)\)

  Con \(D_1\oplus\cdots\oplus D_m\) hago lo mismo.
  \[
    M=Ay_1\dot{+}\cdots\dot{+}Ay_n
  \]
  ordenados bajo el mismo criterio.

  El enunciado se sigue de aplicar el teorema anterior. De
  \(\ann(x_i)=\ann(y_i)\) se deduce
  \[
    C_i\cong Ax_i\cong A/\ann(x_i)=A/\ann(y_i)\cong Ay_i\cong D_i
  \]
\end{proof}

Ejercicio: Decimos que un módulo \(M\) es indescomponible si \(M\cong
L\oplus N\) implica que \(L=\{0\}\) (o \(N=\{0\}\)).
Razonar que en el corolario cada uno de los \(C_i\) es indescomponible.

Ejemplo: \(M\) grupo abeliano de longitud finita y \(p\)-primario.
Aplicando el corolario, \(M\cong C_1\oplus\cdots\oplus C_n\) con
\(C_i\) cíclico y de longitud finita \(p\)-primarios.
Tenemos que \(M\cong \Z_{p^{m_1}}\oplus\cdots\oplus\Z_{p^{m_n}}\),
\(M\) es finito de cardinal \(p^{m_1+\cdots+m_n}\).

\begin{teo}
  \(\subscriptbefore{A}{M}\neq\{0\}\) de longitud finita. Existen
  irreducibles distintos \(p_1,\ldots,p_r\in A\) y enteros positivos
  \(n_1,\ldots, n_r\), tales que \(e_{i1}\ge\ldots\ge e_{in_i}\) con
  \(i\in\{1,\ldots,r\}\) determinados por M:\@
  \[
    M=\dot{+}_{i=1}^r\left(\dot{+}_{j=1}^{n_i} Ax_{ij}\right)
  \]
  para \(x_{ij}\in M\) adecuados que verifican:
  \[
    \ann_A(x_{ij})=\langle p_i^{e_{ij}}\rangle
  \]
  con \(i\in\{1,\ldots,r\}, j\in\{1,\ldots, n_i\}\). Estos parámetros
  determinan \(M\) salvo isomorfismos.
\end{teo}

\begin{proof}
  Supongamos otra descomposición:
  \[
    M=N_1\dot{+} N_t
  \]
  con \(N_i\) \(s_i\)-primario para \(s_1,\ldots,s_t\in A\) irreducibles.
  Entonces
  \[
    \left\langle \mu\right\rangle =\Ann_A(M)=\bigcap_{i=1}^t \Ann_A(N_i)
    =\bigcap_{i=1}^t\left\langle s_i^{t_i}\right\rangle
    = \left\langle\mcm\{s_i^{t_i}\}\right\rangle=\left\langle\prod s_i^{t_i}\right\rangle
  \]
  y \(\mu\) es asociado con \(s_1^{t_1}\cdots s_t^{t_t}\).
  Tras reordenación, por ser \(A\) un DFU, \(t=r\) y \(s_i=p_i\).

  \(N_i\subseteq\{m\in M:p_i^{e_i} m=0\}=M_i\), entonces
  \(N_i=M_i\), argumentando sobre las longitudes.

\end{proof}

\begin{obs}
  Sea \(M\) un grupo abeliano de longitud finita, \(A=\Z\).
  Los grupos abelianos son de longitud finita si y solo si son
  finitos.
\end{obs}
\begin{proof}
  \(\mu=p_1^{e_1}\cdots p_r^{e_r}\)
  \[
    M=\dot{+}_{i=1}^r\dot{+}_{j=1}^{n_i}\Z x_{ij}\cong
    \oplus_{i=1}^r\oplus_{j=1}^{n_i}\Z_{p_i^{e_{ij}}}
  \]
  con \(x_{ij}\). Luego es finito de cardinal:
  \[
    m=\prod_{i=1}^r\prod_{j=1}^{n_i} p_i^{e_{ij}}
    =p_1^{f_1}\cdots p_r^{f_r}
  \]
  donde \(f_i=\sum_{j=1}^{n_i} e_{ij}\).

  \(\mu| m\).

\end{proof}

Ejemplo: si \(m=12\), \(p_1=2\) y \(p_2=3\). Entonces \(M\cong \Z_4\oplus
\Z_3\cong\Z_{12}\) o \(M\cong \Z_2\oplus
\Z_2\oplus\Z_3\cong\Z_2\Z_{6}\).

Ejemplo: \(A=K[x]\) y \(V\) un \(K[x]\)-módulo de longitud finita.
\(V\) es dimensión finita:
\[
  V=\dot{+}_{i=1}^r\dot{+}_{j=1}^{n_i} K[x]x_{ij}
\]
luego es suma directa de espacios de dimensión finita.

\[
  V_{ij}=K[x]x_{ij}\subseteq V
\]
donde \(T(V_{ij})\subseteq V_{ij}\). Tenemos que
\[
  \minpol(\left.T\right|_{V_{ij}})=p_i^{e_{ij}}
\]
existen \(x_{ij}\) tales que \(\{x_{ij},Tx_{ij},\ldots,
T^{\dim V -1}x_{ij}\}\) base de \(V_{ij}\).

Caso particular: \(\dim V=n\), \(\minpol(T)={(x-\lambda)}^n\).
Existe un \(v\in V\) tal que
\[
  \{v, (T-\lambda)v,\ldots,{(T-\lambda)}^{n-1} v\}
\]

Aplicamos \(T{(T-\lambda)}^i v=(T-\lambda+\lambda){(T-\lambda)}^i v=
{(T-\lambda)}^{i+1}v+\lambda{(T-\lambda)}^i v\).
La matriz asociada es:
\[
  M_B(T)=
  \begin{pmatrix}
    \lambda&1&0&0&\cdots&0\\
    0&\lambda&1&0&\cdots&0\\
    0&0&\lambda&1&\cdots&0\\
    \vdots&\vdots&\vdots&\vdots&\cdots&\vdots\\
    0&0&0&0&\cdots&1\\
    0&0&0&0&\cdots&\lambda\\
  \end{pmatrix}
\]
A matrices de este tipo las llamaremos bloque de Jordan.

Si le aplicamos al caso general en el que
\(\mu={(x-\lambda_1)}^{e_1}\cdots{(x-\lambda_r)}^{e_r}\). Tomamos en
cada \(V_{ij} =K[x]x_{ij}\) la base \(\{x_{ij},\ldots,
{(T-\lambda)}^{e_{ij}-1} x_{ij}\}\) y obtenemos uniendo ordenadamente las
bases una base de \(V\), llámase \(B\), tal que por bloques se expresa:
\[
  M_B(T)=
  \begin{pmatrix}
    J_{e_{ij}}(\lambda_i)&0&0&0&\cdots&0\\
    0&J_{e_{ij}}(\lambda_i)&0&0&\cdots&0\\
    0&0&J_{e_{ij}}(\lambda_i)&0&\cdots&0\\
    \vdots&\vdots&\vdots&\vdots&\cdots&\vdots\\
    0&0&0&0&\cdots&0\\
    0&0&0&0&\cdots&J_{e_{ij}}(\lambda_i)\\
  \end{pmatrix}
\]


\end{document}
