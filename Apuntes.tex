\documentclass[12pt,a4paper]{article}
\usepackage[utf8]{inputenc}
\usepackage[spanish]{babel}
\usepackage[T1]{fontenc}
\usepackage{amsmath}
\usepackage{amsthm}
\usepackage{amsfonts}
\usepackage{amssymb}
\usepackage{graphicx}
\usepackage{physics}
\usepackage{mathtools}
\usepackage{hyperref}
\usepackage{xcolor}
\hypersetup{
  colorlinks=false,
  allbordercolors=white
}


\theoremstyle{definition}
\newtheorem{df}{Definición}

\theoremstyle{definition}
\newtheorem{cor}{Corolario}

\theoremstyle{definition}
\newtheorem*{nt}{Notación}

\theoremstyle{remark}
\newtheorem{obs}{Observación}

\theoremstyle{plain}
\newtheorem{prop}{Proposición}

\theoremstyle{plain}
\newtheorem{lema}{Lema}

\theoremstyle{plain}
\newtheorem{teo}{Teorema}

\newcommand{\C}{\mathbb{C}}
\newcommand{\R}{\mathbb{R}}
\newcommand{\N}{\mathbb{N}}
\newcommand{\Z}{\mathbb{Z}}
\newcommand{\Q}{\mathbb{Q}}
\newcommand{\Cont}{\mathcal{C}}


\newcommand{\md}[1]{\left|#1\right|}

\newcommand{\superscriptbefore}[2]{\prescript{#1}{}{#2}}
\newcommand{\subscriptbefore}[2]{\prescript{}{#1}{#2}}

\DeclareMathOperator{\mcm}{mcm}
\DeclareMathOperator{\mcd}{mcd}
\DeclareMathOperator{\car}{car}
\DeclareMathOperator{\Ad}{Ad}
\DeclareMathOperator{\Biad}{Biad}
\DeclareMathOperator{\Ring}{Ring}
\DeclareMathOperator{\Ann}{Ann}
\DeclareMathOperator{\ann}{ann}
\DeclareMathOperator{\id}{id}
\DeclareMathOperator{\End}{End}
\DeclareMathOperator{\Map}{Map}

\title{Álgebra Abstracta}
\date{}

\begin{document}

\begin{titlepage}

	\centering
	\scshape
	\vspace*{\baselineskip}
	\rule{\textwidth}{1.6pt}\vspace*{-\baselineskip}\vspace*{2pt}
	\rule{\textwidth}{0.4pt}
	\vspace{0.75\baselineskip}
	{\Huge Álgebra Abstracta\\}
	\vspace{3cm}
	{\LARGE José Antonio de la Rosa Cubero}
	
\end{titlepage}


\tableofcontents
\newpage

%\section{Introducción}
\subsection{Generalidades sobre anillos}
\begin{df}[Anillo]
  Sea \(A\) un conjunto en el que existen dos operaciones
  \(+,\cdot:A\times A\longrightarrow A\) tales que:
  \begin{enumerate}
    \item \((A, +,0)\) es un grupo aditivo (conmutativo):
      \begin{itemize}
        \item \((a+b)+c=a+(b+c)\) para todos \(a,b,c\in A\).
        \item \(a+b=b+a\) para todos \(a,b\in A\).
        \item \(a+0=a\) para todo \(a\in A\).
        \item Para todo \(a\in A\) existe un \(-a\in A\)
          tal que \(-a+a=0\).
      \end{itemize}
    \item \((A, \cdot, 1)\) es un monoide:
      \begin{itemize}
        \item \((ab)c=a(bc)\) para todos \(a,b,c\in A\).
        \item \(a\cdot 1=1\cdot a=a\) para todo \(a\in A\).
      \end{itemize}
    \item Se cumplen las siguientes propiedades distributivas:
      \begin{itemize}
        \item \((a+b)c=ac+bc\) para todos \(a,b,c\in A\).
        \item \(a(b+c)=ab+ac\) para todos \(a,b,c\in A\).
      \end{itemize}
     \end{enumerate}
  \end{df}

  \begin{df}[Idelaes]
    Sea \(A\) un anillo. \(I\subset A\) se dice ideal si cumple las
    siguientes propiedades:
    \begin{itemize}
      \item \(I\) es un subgrupo aditivo de \(A\) (es decir,
        \(I\) es un conjunto no vacío que cumple
        \(b-a\in I\) para todo \(a, b\in I\)).
      \item \(ax, xa\in I\) para todo \(a\in I\) y \(x \in A\).
    \end{itemize}
  \end{df}

  \begin{teo}[Teorema de Isomorfía]
    Sea \(f:A\longrightarrow B\) un homomorfismo de anillos. Entonces:
    \begin{itemize}
      \item \(\ker f\) es un ideal de \(A\),
      \item \(\Im f\) es un subanillo de \(B\),
      \item Si \(I\subset \ker f\) es un ideal de \(A\), entonces
        existe un único homomorfismo de anillos tal que
        \(\tilde{f}:A/I\longrightarrow B\) tal que \(\tilde{f}(a+I)=f(a)\).
      \item El homomorfismo anterior es inyectivo si y solo si
        \(I=\ker f\).
      \item El homomorfismo anterior es sobreyectivo si y solo si lo era
        \(f\).
    \end{itemize}
  \end{teo}


%\begin{df}[Homomorfismo de anillos]
  \(A,B\) anillos. Se dice que \(f:A\longrightarrow B\) se dice un
  (homo)morfismo de anillos si para todos \(a,a'\in A\) se tiene:
  \begin{enumerate}
    \item \(f(a+a')=f(a)+f(a')\)
    \item \(f(aa')=f(a)f(a')\)
    \item \(f(1)=1\)
  \end{enumerate}
\end{df}

La suma de ideales es un ideal.

\begin{df}[Ideales coprimos]
  Dos ideales \(I, J\subset A\) se dirán primos entre sí
  o coprimos si \(I+J=A\).

  Equivalentemente, existen \(x\in I\), \(y\in J\) tales que
  \(1=x+y\).
\end{df}

La motivación de la definición anterior reside en la identidad
de Bezout, que estamos generalizando.

\begin{lema}
  Sean \(I, J, K\) ideales de \(A\),
  \(I+J=I+K=A\) si y solo si \(I+(J\cap K)=I+J\cap K=A\).

  Es decir, son coprimos entre sí si y solo si uno es coprimo
  con la intersección de los otros dos.
\end{lema}
\begin{proof}
  \[1=x+y=x'+z\]
  con \(x,x'\in I\), \(I\in J\), \(z\in K\).
  \[1=x+y=x+y1=x+y(x'+z)=x+yx'+yz\]
  \(x+yx'\in I\), y \(yz\in J\cap K\).

  Para el recíproco, \(A\supseteq I+J\supseteq I+J\cap K=A\),
  luego \(A=I+J\).
\end{proof}

\begin{lema}
  Sean \(I_1,\ldots, I_t\) ideales de \(A\).
  \(I_1\cap I_i=A\) si y solo si
  \(I_1+\bigcap_{i=2}^t I_i=A\).
\end{lema}
\begin{proof}
  Para \(t=2\) es trivial.

  Supongamos cierto \(I_1\cap I_i=A\implies
  I_1+\bigcap_{i=2}^t I_i=A\) para \(t\), veamos para \(t+1\).

  Llamo \(I=I\), \(J=\bigcap_{i=2}^t I_i\), \(K=I_{t+1}\).
  Por hipótesis de inducción \(I+J=A\) y \(I+K=A\) por ser coprimos
  (hipótesis del lema). Por el lema anterior tenemos:
  \[
    I+J\cap K=I_1+I_{t+1}\cap \bigcap_{i=2}^{t}
    I_i =I_1 + \bigcap_{i=2}^{t+1} I_i
  \]

  La otra implicación es muy sencilla.
\end{proof}

  Hipótesis de trabajo para el teorema chino del resto:
  \begin{enumerate}
    \item \(A\) un anillo.
    \item \(A_1,\ldots,A_t\) anillos.
    \item \(f_i:A\longrightarrow A_i\) un homomorfismo de anillos
      para cada \(i\in\{1,\ldots,t\}\).
    \item \(\Im f_i\subseteq A_i\) es un subanillo.
    \item A \(\Im f_1\times\cdots\times\Im f_t\) se le llama el anillo
      producto.
    \item Definimos \(f:A\longrightarrow\Im f_1\times\cdots\times\Im f_t\),
      \(f(x)=(f_1(x),\ldots,f_t(x))\) para cada \(x\in A\).
    \item Tenemos que \(f\) es un homomorfismo de anillos, cuyo núcleo es la
      intersección de todos los núcleos. Llamaremos \(I=\ker f\).
      \(x\in A\), \(x\in\ker f\) si y solo si \(f_i(x)=0\) para todo \(i\),
      es decir, \(x\in\bigcap_{i=0}^t \ker f_i\).
    \item  Además, existe \(\tilde{f}:A/I
      \longrightarrow\Im f_1\times\cdots\times\Im f_t\),
      con \(x+I\mapsto f(x)\).
    \item Cada \(\ker f_i\) es coprimo con cualquier \(\ker f_j\)
      para \(j\neq i\).
    \item Llamamos \(I_i=\ker f_i\).
  \end{enumerate}

\begin{teo}[Teorema Chino del Resto]
  \(\tilde{f}\) es isomorfismo si y solo si
  \(I_i+I_j=A\) para todo \(i\neq j\).
\end{teo}
\begin{proof}
  Veamos primero la implicación a la derecha.

  Vamos a suponer \(\tilde{f}\) sobreyectiva, es decir, que \(f\) lo es.
  Veamos que todos los \(I_i\) son coprimos entre sí.

  Dado \(i\) tomamos \(x\in A\) tal que \(f_i(x)=1\) y
  \(f_i(x)=0\) para todo \(j\neq i\).

  Observemos que \(x-1\in I_i\), \(x\bigcap_{j\neq i} I_j\)
  \[
    1=1-x+x\in I_i+\bigcap_{j\neq i} I_j
  \]

  Por tanto, \(I_i+\bigcap I_j =A\) y entonces por el lema anterior
  \(I_i+I_j=A\).

  Veamos el recíproco.
  Suponemos que \(I_i+I_j=A\) para cualquier \(i\neq j\).

  Tomamos \((f(b_1),\ldots,f(b_t))\in I_1\times\cdots\times I_t\).

  Para cada \(i\), tomamos
  \(
  1=a_i+p_i\) con \(a_i\in I_i\) y \(p_i\in\bigcap I_j\).

  Tomamos \(x=\sum_{i=1}^t b_i p_i\).

  \[
    f_j(x)=\sum_{k=1}^t f_j(b_k) f_j(p_k)=f_j(b_j)f_j(p_j)=f_j(b_j(1-a_j))
    =f_j(b_j)-f_j(b_j)f_j(a_j)=f_j(b_j)
  \]
  porque \(f_j(p_k)=0\) si \(k\neq j\) y \(a_j\in\ker f_j\).

\end{proof}

%
\begin{obs}
  Para anillos conmutativos denotamos
  \[
    \langle a\rangle=\{ba:b\in A\}
  \]
  el ideal generado por \(a\).
\end{obs}

Vamos a hacer un ejemplo, aplicando el teorema anterior.

\subsubsection{Interpolación}         

Tomamos \(A=K[x]\), un anillo de polinomios con coeficientes en un
cuerpo \(K\).

Sea \(A_i = K\) con \(i\in\{1,\ldots,t\}\).
Tomamos \(\alpha_i\in K\) para cada \(i\) y definimos
\(\xi_i:K[x]\longrightarrow K\), \(\xi(g)=g(\alpha_i)\),
para cada \(g\in K[x]\) y es un homeomorfismo de anillos.

\(\Im X_i=K\) y \(\xi:K[x]\longrightarrow K\times
\cdots K=K^t\).

\(\ker \xi_i=\langle x-\alpha_i\rangle\) que es ideal de un anillo de
polinomios, luego principal. Está generado por el polinomio de grado menor,
como las constantes no pueden anular a \(\xi_i\), tiene que estar generado
por ese, que es de grado uno.

\[
  I=\bigcap_{i=1}^t\langle x-\alpha_i\rangle =\langle p(x)\rangle
\]
donde \(p(x)=\mcm\{x-\alpha_i: i\in\{1,\ldots, t\}\}\).


El teorema chino del resto nos asegura que \(\tilde{\xi}:
K[x]/\langle p(x)\rangle\longrightarrow K^t\) es un isomorfismo si y solo si
\(\mcd\{x-\alpha_i, x-\alpha_j\}\) para todo \(j\neq i\), es decir,
si \(\alpha_i\neq \alpha_j\).

Lo que estamos viendo es que para cualquier tupla
\((y_1,\ldots, y_t)\in K^t\), existe un \(g\in K[x]\) tal que
\(g(\alpha_i)=y_i\), si y solo si \(\alpha_i\neq\alpha_j\). En tal
caso, \(p(x)=\prod_{i=1}^t(x-\alpha_i)\).

Existe un único representante \(g\in K[x]\) tal que \(g(\alpha_i)=y_i\)
de grado menor que \(t\), siempre que
\(p(x)=\prod_{i=1}^t(x-\alpha_i)\).

\(\alpha_1,\ldots,\alpha_t\in K\) distintos dos a dos
\[
  \tilde{\xi}:K[x]/\langle p(x)\rangle\longrightarrow K^t
\]
es un isomorfismo de anillos.

\(K[x]/\langle p(x)\rangle\) es un espacio vectorial cociente.

\(\tilde{\xi}\) es también un isomorfismo entre espacios vectoriales.

\[
  \tilde{\xi}(\alpha(g+p))=\tilde{\xi}(\alpha g+p)=\tilde{\xi}((\alpha
  + p)(g +p))=\]\[\tilde{\xi}(\alpha + p)\tilde{\xi}(g+p)=
  (\alpha,\ldots, \alpha)(g(\alpha_1),\ldots g(\alpha_t))=
  \alpha\tilde{\xi}(g+p)
\]

Sea \(\{1+p, x+p, x^2+p,\ldots, x^{t-1} +p\}\)
\(K\)-base de \(K[x]/\langle p(x)\rangle\).
Notamos:
\[ 1 = 1+p\]
\[ x = x+p\]

Sea \(\{e_1,\ldots, e_n\}\) es la base canónica de \(K^t\).
Nuestro objetivo es calcular sus preimagenes por \(\xi\), en concreto
un polinomio de grado menor que \(t\).

\[
  g_i(x)=\prod_{j\neq i}(x-\alpha_j)
\]

\[
  L_i(x)=\frac{g_i(x)}{g(\alpha_i)}=\prod_{j\neq i}\frac{x-x_j}{x_i-x_j}
\]

que vale 0 en \(\alpha_j\) para cualquier \(j\)
salvo en \(\alpha_i\) que vale 1.

Tenemos que
\[
  g(x)=\sum_{i=1}^t y_i L_i(x)
\]
satisface que \(g(\alpha_i)=y_i\).


Finalmente vamos a ver que la matriz de \(\tilde{\xi}\) en las bases
consideradas es:
\[
  \begin{pmatrix}
    1&\cdots& 1\\
    \alpha_1 &\cdots&\alpha_t\\
    \cdots&\cdots&\cdots\\
    \alpha_1^t&\cdots&\alpha_t^t
  \end{pmatrix}
\]

%\subsubsection{Transformada discreta de Fourier}

Ahora vamos a reindexar. En lugar de usar \(1,\ldots, t\)
vamos a tomar los índices \(1,\ldots, n-1\).

Vamos a suponer que el cuerpo \(K\)
contiene una raíz primitiva de 1, o sea,
existe un \(\omega\in K\) tal que \(\omega^n = 1\) y
\(1, \omega, \omega^2,\ldots, \omega^{n-1}\) son distintos.

Seguro que \(\car K \not\| n\) ya que \(1, \omega, \omega^2,\ldots,
\omega^{n-1}\) son las raíces de \(x^n-1\) y son distintas.

Vamos a interpolar las raíces de la unidad.

Tomo \(\alpha_j =\omega^j\), \(j\in\{0,\ldots, n\}\) y
\[M=A_\omega=
\begin{pmatrix}
  1&1&\cdots1\\
  \omega^0&\omega^1&\cdots\omega^{n-1}\\
  {(\omega^0)}^2&{(\omega^1)}^2&\cdots{(\omega^{n-1})}^2\\
  \cdots&\cdots&\cdots
\end{pmatrix}= (\omega^{ij})
\]


Tenemos que \(x^n-1=(x-1)(x^{n-1}+\cdots+x+1)\)
y evaluando en \(\omega^{j}\) obtenemos
\[
  \omega^{(n-1)j}+\cdots+\omega^j+1=0
\]

Entonces \(\sum_{k=0}^{n-1}\omega^{ik}=0\) para \(0<i<n\).

\[
  \begin{pmatrix}
    \omega^{i}&
    \omega^{2i}&
    \cdots&
    \omega^{(n-1)i}
  \end{pmatrix}
  \begin{pmatrix}
    \omega^{-j}\\
    \omega^{-2j}\\
    \cdots\\
    \omega^{-(n-1)j}
  \end{pmatrix}
  =\sum_{k=0}^{n-1}\omega^{k(i-j)}=0
\]

Tenemos entonces que \(A_\omega A_{\omega^{-1}}^T=nI\),
es decir, \(A^{-1}_\omega=\frac{1}{n}A_{\omega^{-1}}^T\).

\(\tilde{\xi}:K[x]/\langle x^n-1\rangle\longrightarrow K^n\),
con \(\xi^{-1}(y)\) es el polinomio interpolador.

Tenemos unos datos \((y_0,\ldots, y_{n-1})\in K^n\). El polinomio
interpolador de esos datos en los nodos \(1,\omega,\ldots,\omega^{n-1}\)
viene dado por
\[
  \hat{y} =\sum_{j=0}^{n-1}\hat{y_j}x^j
\]
donde \(\hat{y}=y\frac{1}{n}A^T_{\omega^{-1}}\).

Explicitamente, se calcula que los coeficientes quedan:
\[
  y_j=\frac{1}{n}\sum_{k=0}^{n-1}y_k\omega^{-jk}
\]

Tomamos \(K=\C\). \(\omega = e^{i2\pi/n}\):
\[
  y_j=\frac{1}{n}\sum_{k=0}^{n-1}y_k\omega^{-i2\pi jk/n}
\]
que es la transformada de Fourier de \(y\).

¿Qué interpretación le damos? Supongamos una función periódica de periodo
\(2\pi\), \(f:[0,2\pi]\longrightarrow\C\) con \(f(0)=f(2\pi)\).
Dividimos el intervalo en \(n\) partes iguales, una muestra:
\(y_j=f(\frac{2\pi j}{n})\) con \(j=0,\ldots,n-1\).

Tomomamos \(g:[0,2\pi]\longrightarrow\C\) con
\(g(t)=\sum_{j=0}^{n-1}\hat{y_j}e^{ijt}\).

Tenemos entonces que \(g(\frac{2\pi l}{n})=
\sum_{l=0}^{n-1}\hat{y_j}e^{i2\pi lj/n} = y_l=f(\frac{2\pi j}{n})\)

A los \(\hat{y}\) también se le llama el espectro de \(y\).

%\section{Módulos}

\begin{df}
  Sean \(M\), \(N\) grupos aditivos:
  \[
    \Ad(M,N)=\{f:M\longrightarrow N| f\textrm{ homomorfismo de grupos}\}
  \]
\end{df}

El conjunto anterior es no vacío porque \(0\in\Ad(M,N)\).
\(\Ad(M,N)\) es un grupo aditivo con la suma:
\[
  (f+g)(m):=f(m)+g(m)\hspace{1cm} \forall m\in M
\]


\begin{df}[Anillo de endomorfismo de \(M\)]
  Definimos directamente \(\End(M):= \Ad(M,M)\).
\end{df}

\begin{prop}
  \((\End(M),+,0,\circ,\id) \) es un anillo.
\end{prop}
\begin{proof}
  Se comprueba que es cerrado para composición. Es obvio que la
  composición es asociativa y tiene como elemento neutro la identidad.

  Finalmente se ve que se cumplen las propiedades distributivas, que
  se siguen de que son homomorfismos.
\end{proof}


\begin{obs}
  Consideramos el grupo \(\{0\}\), es el anillo \(\{0\}\) (anillo
  cero o trivial).

  Si \(M\neq \{0\}\), entonces \(\End(M)\) no es trivial.
\end{obs}

\begin{df}[Módulo]
  Sea \(M\) un grupo aditivo y \(A\) un anillo. Una estructura de
  \(A\)-módulo sobre \(M\) es un homomorfismo de anillos
  \(\rho: A\longrightarrow\End(M)\).
\end{df}

Ejemplo: los números enteros. \(M\) grupo aditivo, \(A=\Z\).
Existe un único \(\chi:\Z\longrightarrow\End(M)\) determinado
por \(\chi(1)=\id_M\), es decir, una única estructura de
\(\Z\)-módulo sobre \(M\) (y su núcleo te da la
característica del anillo).

Ejemplo: cuerpos. Sea \(K\) un cuerpo.
Si \(V\) es un \(K\)-espacio vectorial, definimos \(\rho:K\longrightarrow
\End(V)\), tomamos \(\rho(\alpha):V\longrightarrow V\)
cumpliendo \(\rho(\alpha)(v)=\alpha v\). Trivialmente se cumple que
\(\rho\) es un homomorfismo por la estructura de espacio vectorial de \(V\).
Con lo cual tenemos una estructura de \(K\)-módulo sobre \(V\).
Se puede demostrar el recíproco trivialmente.

\begin{obs}
  Sean  \(X, Y, Z\) conjuntos. \(\Map(X,Y)\) es el conjunto de
  aplicaciones de \(X\) en \(Y\).

  Entonces:
  \[
    \psi:\Map(X\times Y, Z)\longrightarrow\Map(X,\Map(Y,Z))
  \]
  es una biyección dada por \(\psi(f)(x)(y):=f(x,y)\) y
  \(\psi^{-1}(g)(x,y):=g(x)(y)\).
\end{obs}

Ejercicio: comprobar que \(\psi^{-1}\) es realmente la inversa de
\(\psi\).

\begin{obs}
  Sean \(M, N, L\) grupos aditivos.
  \[
    \psi:\Biad(M\times N, L)\longrightarrow\Ad(M,\Ad(N,L))
  \]
  donde \(b\in\Biad(M\times N, L)\) si \(b\) es biaditiva:
  \[
    b(m+m',n)=b(m,n)+b(m',n)
  \]\[
    b(m,n+n')=b(m,n)+b(m,n')
  \]
\end{obs}

Ejercicio, demostrar que la aplicación \(\psi\) es una biyección.

\begin{teo}[Caracterización de módulos]
  Sea \(A\) anillo, \(M\) un grupo aditivo. Sea \(\Ring(A,\End(M))\),
  llamamos \(A\)-módulo a la imagen por \(\psi\) de ese conjunto.
\end{teo}

%\begin{df}
  \[
    \Ring(R,S)=\{\phi:R\longrightarrow S, \phi \textrm{ es homomorfismo
    de anillos}\}
  \]
\end{df}


\begin{prop}
  Dados un grupo aditivo \(M\) y un anillo \(A\), se tiene una
  correspondencia biyectiva entre:
  \begin{enumerate}
    \item Homomorfismos de anillos \(\rho:A\longrightarrow\End(M)\)
    \item Las aplicaciones \(A\times M\longrightarrow M\)
      que satisfacen:
      \begin{itemize}
        \item \((a+a')m=am+a'm\)
        \item \(a(m+m')=am+am'\)
        \item \((aa')m=a(a'm)\)
        \item \(1\cdot m=m\)
      \end{itemize}
  \end{enumerate}
\end{prop}

\begin{proof}
  Tomamos la biyección \(\psi^{-1}:\Map(A,\Map(M,M))\longrightarrow
  \Map(A\times M,M)\).
  Tomamos \(\rho\in\Ring(A,\End(M))\), su imagen por la biyección,
  \(\psi^{-1}(\rho)\)
  son las aplicaciones que satisfacen justo las propiedades
  anteriores.

  Llamamos a \(\psi^{-1}(\rho)(a,m)=a\cdot m\). Tenemos que
  \(\psi^{-1}(\rho)(a,m)=\rho(a)(m)\). Entonces
  \(a\cdot m=\rho(a)(m)\).

  Comprobamos la tercera propiedad como ejemplo:

  Dados \(a, a'\in A\) y \(m\in M\):
  \[
    (aa')m=\rho(aa')(m)=(\rho(a)\circ\rho(a'))(m)
    =\rho(a)(\rho(a')(m))=\rho(a)(a'm)=a(a'm)
  \]

  De forma análoga se demuestran el resto de propiedades.

  Esta correspondencia responde a la fórmula \(am=\rho(a)(m)\).
\end{proof}

Un \(A\)-módulo lo veré de cualquiera de las maneras anteriores, que
ya hemos visto que son equivalentes, según su conveniencia.

Ejemplo, si \(K\) es un cuerpo, un \(K\)-módulo es esencialmente
un \(K\) espacio vectorial.

Otro ejemplo, el \(A\)-módulo regular. \(A\) es un \(A\)-módulo, vía
\(\lambda:A\longrightarrow(A)\) que lleva cada
\(a\) a \(\lambda(a)(a'):=aa'\). La demostración es sencilla
usando la segunda definición.

\begin{prop}[Restricción de escalares]
  Sea \(\phi:\R\longrightarrow S\) homomorfismo de anillos. Si \(M\)
  es un \(S\)-módulo, vía un homomorfismo de anillos \(\rho:
  S\longrightarrow\End(M)\), tenemos que \(M\) es un
  \(R\)-módulo vía \(\rho\circ\phi\).

  Equivalentemente, si \(r\in R\) y \(m\in M\), definimos
  \[
    rm=(\rho\circ\phi)(r)(m)=\rho(\phi(r))(m)=\phi(r)m
  \]
\end{prop}


%\subsection{\(K[x]\)-módulos con \(K\) cuerpo}

Tenemos \(K[x]\)-módulo \(M\). O sea, \(M\) es un grupo aditivo
y \(\rho: K[x]\longrightarrow\End(M)\) es un homomorfismo de anillos.

\(K\) se puede ver como subanillo de \(K[x]\), aplicando la
restricción de escalares aplicada a la aplicación inclusión,
\(M\) es un \(K\)-espacio vectorial.

Veamos que ocurre con la indeterminada. \(\rho(x)\in\End(M)\).

Veamos que es un endomorfismo de espacios vectoriales:
\[
  \rho(x)(km)=x\cdot (km)=x\cdot(k\cdot m)=(xk)\cdot m
  =kx\cdot m=k(xm)=k\rho(x)(m)
\]

Así que \(\rho(x)\) es \(K\)-lineal.

Si \(p=\sum_i p_i x^i\in K[x]\), tenemos que
\[
  pm=\rho(p)(m)=\sum_i p_i {\rho(x)}^i(m)
\]

\begin{prop}
  Si tengo un \(K\)-espacio vectorial \(V\) y una aplicación
  lineal \(T:V\longrightarrow V\), podemos definir para \(p\in K[x]\)
  y \(v\in V\) el operador
  \[
    pv:=p(T)(v)=\sum_i p_i T^i(v)
  \]
  resulta que \(V\) es un \(K[x]\)-módulo.
\end{prop}

Ejemplo, \(\Cont^\infty(\R)\) con \(T=\frac{d}{dt}\)
es un \(\R[x]\)-módulo.

%\begin{obs}
  \(\Cont^\infty(\R)\) dotado de estructura de \(\R[x]\)-módulo
  a través del endomorfismo lineal \(T=\frac{d}{dt}\) es un ejemplo
  ilustrativo en el siguiente sentido.

  Tomemos \(\sin\), \(x\sin t=T(\sin t)=\cos t\)
  \(x^2\sin t= -\sin t\) con lo que
  \[
    (x^2+1)\sin t=0
  \]
  es decir, en un \(A\)-módulo \(M\) puede pasar que \(a m=0\)
  \(a\neq 0\), \(m\neq 0\).
\end{obs}

Ejemplo en el \(\Z\)-módulo \(\Z_4\) tenemos que
\(2\cdot \bar{2}=\bar{0}\).

%\subsubsection{Módulos abstractos}

Sea \(A\) un anillo, \(M_A\) un \(A\)-módulo,
entonces si tenemos
un homomorfismo de anillos \(\varphi:A\longrightarrow\End(M)\)
cuyo núcleo es un ideal de \(A\).

Aplicando el primer teorema de isomorfía, tenemos:
\[
  A/\ker\varphi\longrightarrow\Im\varphi\subseteq\End(M)
\]
y entonces \(M\) es un \(A/\ker\varphi\)-módulo.
De hecho \((a+\ker\varphi)m=\varphi(m)\).

\[
  \ker\varphi=\{a\in A: am=0\}=\Ann_A(M)
\]
se le llama el anulador de \(M\).

Tenemos que \(M_A\) entonces \(M_{A/\Ann_A(M)}\)

%
\begin{proof}
  Veamos que 1 implica 2. Tenemos que \(0=n_1+\cdots+n_t\),
  si despejamos, \(n_i=-\sum_{j\neq i} n_j\in N_i\cap\left(
  \sum_{j\neq i} N_j\right)=\{0\}\).

  Veamos que 2 implica 3. Si \(n=\sum n_i=\sum n_i'\),
  entonces \(0=\sum(n_i-n_i')\) lo que implica que
  \(n_i=n_i'\).

  Finalmente, tomando \(n\in N_i\cap\left(
  \sum_{j\neq i} N_j\right)\), es decir,
  \(n=\sum_{j\neq i}n_j\) con lo que
  \(0=n-\sum_{j\neq i} n_j\) y como las descomposiciones
  son únicas, \(n=0\).
\end{proof}

\begin{df}[Suma interna]
  Si \(M=N_1+\cdots+ N_t\) tales que satisfacen una de las condiciones
  equivalentes anteriores, diremos que \(M\) es la suma directa interna
  y usaremos la notación \(M=N_1\dot{+}\cdots\dot{+} N_t\).
\end{df}

\begin{df}
  Si \(\{N_1,\ldots, N_t\}\) verifican las condiciones equivalentes
  anteriores y \(N_i\neq \{0\}\), se dice que el conjunto
  \(\{N_1,\ldots, N_t\}\) es una
  familia independiente.
\end{df}

Ejemplo: \(\Z_6\) es un \(\Z\) módulo.
\[
  \Z_6=\{0,1,2,3,4,5,6\}
\]
Tomamos
\[
  N_1=\{0,3\}
\]
y
\[
  N_2=\{0,2,4\}
\]

Tenemos que \(N_1, N_2\) es una familia independiente. Además es obvio que:
\[
  N_1\dot{+}N_2=\Z_6
\]
ya que tienen como intersección \(\{0\}\) y su suma es el total.


%\subsection{Módulos acotados sobre un DIP}

\begin{df}[Módulo acotado sobre un DIP]
  Sea \(A\) un dominio de ideales principales, \(\subscriptbefore{A}{M}\)
  un módulo, \(\Ann_A(M)=\langle\mu\rangle\) para cierto \(\mu\in A\).

  Si \(\mu\neq 0\), diré que \(M\) es acotado.
\end{df}

Supongamos que \(\subscriptbefore{A}{M}\) es acotado y \(\mu\not\in
\mathcal{U}(A)\), ya que si \(\mu\in\mathcal(A)\) entonces \(M=\{0\}\).

Si \(\mu=p_1^{e_1}\cdots p_t^{e_t}\), posible porque todo DIP es un
dominio de factorización única (DFU), con \(p_i\in A\) irreducible
y \(e_i>0\).

\begin{prop}[Descomposición primaria del módulo]
  Tomamos \(q_i=\frac{\mu}{p_i^{e_i}}\in A\).

  Llamamos \(M_i=\{q_i m:m \in M\}\subseteq M\). Veamos
  que \(M_i\in\mathcal{L}(\subscriptbefore{A}{M})=\{\textrm{submódulos
  de } \subscriptbefore{A}{M}\}\).

  Queremos que \(M=M_1\dot{+}\cdots \dot{+}M_t\), con \(t>1\) para
  evitar trivialidades. En ese caso, \(\mcd\{q_1,\ldots,q_t\}=1\),
  donde se ha usado que estamos en un DFU.\@

  Por la identidad de Bezout (válida porque estamos en un DIP),
  tenemos que \(1=\sum_{i=1}^t q_i a_i\), para ciertos \(q_i\in A\).
  Para en \(m\in M\), \(M=1\cdot m=\sum_i q_i a_i m\), luego
  \(M=M_1+\cdots+ M_t\).

  Vamos a ver que la suma es directa.
  \(q_i q_j\in\langle\mu\rangle\) si \(i\neq j\). Eso significa que si
  \(m\in M_i\) y entonces \(q_j m= 0\) si \(i\neq j\).
  Por tanto \(M_i=\{m\in N: m=q_i a_i m\}\).

  Si \(0=\sum_{i=1}^t\) con \(m_i\in M_i\), entonces
  \[
    0=q_j a_j 0=m_j
  \]
  y por tanto \(M=M_1\dot{+}\cdots\dot{+} M_t\).
\end{prop}

\begin{df}[Componentes primarias]
  Tenemos que los \(M_i\) se llaman componentes primarias.
\end{df}
\begin{prop}
  \[
    M_i=\{m\in M: p_i^{e_i} m=0\}
  \]

  Así, \(\langle\mu\rangle=\Ann_A(M)=\bigcap_{i=1}^t\Ann_A(M_i)
  \supseteq\bigcap_{i=1}^t\langle p_i^{e_i}\rangle=\langle\mu\rangle\)
\end{prop}

Ejercicio: Obtener la descomposición primaria usando \(\dot{+}\) de
\(\Z_{8000}\).

%Ejemplo: \(T\) endomorfismo \(K\)-lineal. \(V=\subscriptbefore{K[x]}{V}\).

Un \(W\) es un submódulo de \(V\) es un subespacio vectorial tal que
\(T(W)\subseteq W\), es decir, \(W\) es \(T\) invariante.

Si \(\Ann_{K[x]}(V)\neq\{0\}\), tomo \(\mu(x)\in K[x]\), el polinomio
mínimo de \(T\). Es decir, \(\Ann_{K[x]}(V)=\langle\mu(x)\rangle\).

\[
  \mu=p_1^{e_1}\cdots p_t^{e_t}
\]

Entonces la descomposición primaria de \(V\) es \(V=V_1\dot{+}
\cdots\dot{+}V_t\) con
\[
  V_i=\{v\in V:p_i(x)v=0\}
\]

Caso particular: \(\dim(V)<\infty\) y que \(\mu(x)=(x-\alpha_1)\cdots
(x-\alpha_t)\) con \(\alpha_i\neq\alpha_j\).
\[
  V_i=\{v\in V:(x-\alpha_i)v=\{v\in V:T(v)=\alpha_i v\}
\]
es decir, el subespacio propio asociado al valor propio \(\alpha_i\).

Si el polinomio factoriza como producto de polinomios de grado 1 distintos,
\(T\) es diagonalizable.
Veremos en el futuro que el polinomio mínimo divide siempre al polinomio
característico.

¿Cómo se calcula el polinomio mínimo de un endomorfismo lineal?

Ejercicio: Sea \(V\) un espacio vectorial real euclídeo (con producto
escalar). Sea \(T:V\longrightarrow V\) una isometría.
Se pide demostrar que si \(W\) es un subespacio \(T\) invariante de \(V\),
entonces su ortogonal \(W^\perp\) es también \(T\) invariante.
Entonces \(V=W\dot{+}W^\perp\). Se usa inducción. Como consecuencia, usando
el teorema fundamental del álgebra, deducir que \(V\) admite una base
ortonormal con respecto de la cual la matriz de \(T\) es diagonal por
bloques, con bloques de dimensión 1 o 2. ¿Qué aspecto tienen dichos
bloques? Hay que ver que uno de los dos subespacios invariantes tienen
dimensión 1 o 2.


%\subsection{Homomorfismos de módulos}

\begin{df}[Módulo cociente o factor]
  Sea \(\subscriptbefore{A}{M}\) y \(L\in\mathcal{L}(M)\).
  Consideramos \(M/L\) grupo aditivo y se define la acción:
  \[
    a(m+L):=am+L
  \]
  \(M/L\) es un módulo.
\end{df}

\begin{df}[Homomorfismo de módulos]
  Se dice que
  \(f:\subscriptbefore{A}{M}\longrightarrow \subscriptbefore{A}{N}\)
  es un homomorfismo de módulos si respeta sumas y productos.
\end{df}

\begin{df}[Proyección canónica]
  Es la aplicación \(\pi:M\longrightarrow M/L\) dada por
  \(\pi(m)=m+L\) es un homomorfismo de módulos.
\end{df}

\begin{teo}[Teorema de isomorfía para módulos]
  \(f:M\longrightarrow N\) un homorfismo de \(A\)-módulos. Entonces
  el núcleo \(\ker f\in\mathcal{L}(\subscriptbefore{A}{M})\) y
  \(\Im f\in\mathcal{L}(N)\). Para cada \(L\in\mathcal{L}
  (\subscriptbefore{A}{M})\) tal que \(L\subseteq \ker f\) existe
  un único homomorfismo de módulos \(\tilde{f}:M/L\longrightarrow N\)
  tal que \(\tilde{f}\circ\pi=f\). Finalmente, \(\tilde{f}\) es
  inyectiva si y solo si \(L=\ker f\), en cuyo caso, \(\tilde{f}\)
  da un isomorfismo de \(A\)-módulos
  \(M/\ker f\cong \Im f\).
\end{teo}

Ejemplo \(\subscriptbefore{A}{M}\), definimos \(f:A\longrightarrow M\)
dada por:
\[
  f(a)=am\hspace{1cm} \forall a \in A
\]
es un homomorfismo de \(A\)-módulos.

Tenemos \(\Im f = Am\) y
\(\ann(a)=\ker f=\{a\in A: am=0\}\) es un ideal izquierda y se tiene
\[
  A/\ann_A(m)\cong Am
\]


%\[
  a+\ann_A(m)\mapsto am
\]

Ejemplo: \(S=\Map(\N,K)\), el conjunto de las sucesiones (que forman
un \(K\)-espacio vectorial). Tomamos \(T:S\longrightarrow S\)
tal que \(T(s)(n)=s(n+1)\). Es lineal. Entonces
\(\subscriptbefore{K[x]}{S}\), donde \(xs=T(s)\).

Para cualquier \(f\in K[x]\), es decir \(f=\sum_i f_i x^i\), se tiene:
\[
  (fs)(n)=\sum_i f_i s(n+i)
\]

Imaginémosnos que \(s\) verifica que \(\ann_{K[x]}(s)\neq\langle0\rangle\).
Podemos tomar entonces un polinomio tal que \(fs = 0\) y que sea mónico.
Tenemos entonces que \(s(n+m)=-\sum_{i=0}^{m-1} f_i s(n+i)\)
para todo \(n\in\N\). Es decir, la sucesión es linealmente recursiva.

Caso particular, \(s(0)=s(1)=1\), tenemos que
\[
  s(n+2)=s(n)+s(n+1)
\]

\[
  x^2-x-1\in\ann_{\Q[x]}(s)
\]

Volviendo al caso general, tenemos que
\[
  K[x]/\ann_{K[x]}(s)\cong K[x]s
\]

Tenemos que \(\dim_{K}(K[x]s)<\infty\) si y solo si
\(\ann_{K[x]}(s)\neq\langle0\rangle\) si y solo si
\(s\) es una sucesión linealmente recursiva.

El generador \(p(x)\) de \(\ann_{K[x]}(s)\) se le llama el polinomio
mínimo de \(s\). El grado de dicho polinomio, coincide con
\(\dim_{k}(K[x]s)\) y se le llama complejidad lineal de \(s\).

\(s,t\) dos sucesiones linealmente recursivas.
\(K[x](s+t)\subseteq K[x]s+K[x]t\), luego la primera tiene dimensión finita.
Luego \(s+t\) es una sucesión linealmente recursiva, de complejidad menor
o igual a la suma de las complejidades lineales.
Puede argumentarse lo mismo para combinaciones lineales.

Las sucesiones linealmente recursivas forman un subespacio vectorial
del espacio de sucesiones. De hecho forman un submódulo. Sea
\(S^l\) el conjunto de las sucesiones linealmente recursivas, forma
un \(S^l\) es un \(K[x]\)-submódulo de \(S\), ya que es ivariante por la
acción de \(x\) (es \(T\)-invariante).

Otro ejemplo: \(T\) endomorfismo de \(\Cont^\infty(\R)\) tal que
\(T(\varphi)=\varphi'\). Tenemos que
\(\subscriptbefore{R[x]}{\Cont^\infty(\R)}\). Dada \(\varphi\),
tenemos que
\[
  \ann_{\R[x]}(\varphi)=\{f\in\R[x]: f(x)\varphi=0\}
  =\{f=\sum_i f_i\frac{d^i}{dt^i}: f\varphi=0\}
\]
\(\ann(\varphi)\neq\langle 0\rangle\) si \(\varphi\) satisface una ecuación
diferencial lineal homogénea con coeficientes constantes. Bla bla.

\(\R[x]/\ann_{\R[x]}(\varphi)\cong \R[x]\varphi\), donde \(\varphi\)
satisface bla bla.

Tenemos que \(\varphi''-\varphi'-\varphi=0\), cuya solución
\(\varphi(t)=e^{\phi t}\), donde \(\phi\) es la razón aúrea.


%\subsubsection{Suma directa externa}

\begin{df}
Tomando el producto cartesiano de \(t\) módulos sobre el mismo anillo
y tomando la suma usual de tuplas y definiendo el siguiente producto:
\[
  a(m_1,\ldots, m_t)=(am_1,\ldots,am_t)
\]

Es un módulo que se llama suma directa externa de \(M_1,\ldots, M_t\)
con \(M^t\) si son todos iguales.

  Se denota \(M_1\oplus\cdots\oplus M_t\).
\end{df}


Ejercicio: Sea \(\subscriptbefore{A}{M}\), \(N_1,\ldots,
N_t\in\mathcal{L}(\subscriptbefore{A}{M})\). Se pide demostrar que existe
un homomorfismo \(f:N_1\oplus\cdots\oplus N_t\longrightarrow
N_1{+}\cdots{+}N_t\)
sobreyectivo de \(A\)-módulos tal que entre la suma directa
externa y la suma interna, tal que \(f\) es un isomorfismo si y solo si
la suma interna es directa. Podría ser interesante usar coordenadas.

%\begin{df}[Base de un módulo libre]
  Consideramos \(A^n=A\oplus\cdots\oplus A\), donde la suma se repite
  \(n\) veces. Para cada \(i=1,\ldots, n\), tenemos que
  \(\{e_i: e_i=(0,\ldots, 0,1,0, \ldots, 0)\}\) forman un sistema de
  generadores de \(A^n\). Por tanto \(a=\sum_i a_i e_i\in A^n\)
  es una expresión única.
\end{df}

Dicha base puede no existir.

\begin{prop}
  Dado un módulo cualquiera \(\subscriptbefore{A}{M}\) y \(m_1, m_n\in M\),
  existe un único homomorfismo de módulos \(f:A^n\longrightarrow M\)
  tal que \(f(e_i)=m_i\).
\end{prop}

\begin{cor}
  Si \(M\) es finitamente generado con generadores \(\{m_i\}\),
  entonces \(M\cong A^n/L\) para \(L\) un sierto submódulo.
\end{cor}

\begin{proof}
  Unicidad: si existe una tal aplicación \(f\), entonces para
  cualquier \(a\in A^n\),
  \[
    f(a)=\sum_i a_i f(e_i)=\sum_i a_i m_i
  \]

  Veamos la existencia,
  Definiendo \(f(a)=\sum_i a_i m_i\) obtenemos un homomorfismo de módulos
  que cumple lo exigido en el enunciado.

  Si \(M=Am_1+\cdots+Am_n\) tenemos que \(L=\ker f\) cumple lo que se
  pide por el teorema de isomorfía para módulos.
\end{proof}


%\section{Módulos Noetherianos}
\subsection{Álgebra homológica}

\begin{df}[Sucesiones exactas]
  Una suceión de homomorfismos de módulos \(f_i:M_i\longrightarrow
  M_{i+1}\) se dice exacta en
  \(M_{i+1}\) si \(\ker f_{i+1}=\Im f_i\).
\end{df}

Ejemplo: Dada una sucesión \(\{0\}\longrightarrow L \alpha
\longrightarrow M \beta\longrightarrow N\longrightarrow \{0\}\)
es exacta en \(L\) si y solo si \(\ker \alpha=\{0\}\), es decir,
\(\alpha\) es inyectiva, en \(N\) si y solo si \(\Im \beta = N\),
es decir, \(\beta\) sobreyectiva y en \(M\) si y solo si
\(\ker\beta=\Im\alpha\).

A \(\alpha\) se les llama monomorfismos de módulos y a
\(\beta\) epimorfismos de módulos.

A esta sucesión se le llama sucesión exacta corta.

Caso particular: Por ejemplo, si \(f:M\longrightarrow N\) es
un homorfismo de módulos, obtenemos:
\[
  0\longrightarrow\ker f\iota\longrightarrow M f\longrightarrow\Im f
  \longrightarrow 0
\]

\begin{prop}
  Sea \(0\longrightarrow L\overset{\psi}{\longrightarrow} M
  \overset{\varphi}{\longrightarrow}
  N\longrightarrow 0\) una sucesión exacta de \(A\)-módulos. Entonces:
  \begin{enumerate}
    \item Si \(M\) es finitamente generado, lo es también \(N\).
    \item Si \(L\) y \(N\) son finitamente generados, lo es también \(M\).
  \end{enumerate}
\end{prop}

\begin{proof}
  Veamos primero la primera afirmación. Sea \(\{m_i\}\) generadores de \(M\).
  Es claro que \(\{\varphi(m_i)\}\) generan \(N\).

  Para la segunda, \(\{n_i\}\) generadores de \(N\), y tomamos
  \(\{m_i\}\subseteq M\) tales que \(\varphi(m_i)= n_i\).

  Tomamos \(\{e_i\}\) generadores de \(L\). Tomamos \(m\in M\).
  \[
    \varphi(m)=\sum_{i=1}^s r_i n_i = \sum_{i=1} r_i\varphi(m_i)
    =\varphi\left(\sum r_i m_i\right)
  \]
  con lo que \(m-\varphi(\sum r_i m_i)\in\ker\varphi=\Im\psi\).
  Luego existen \(b_1,\ldots, b_t\) tales que
  \[
    m-\varphi\left(\sum r_i m_i\right)=
    \psi\left(\sum_j b_j e_j\right)
  \]
  y finalmente:
  \[
    m=\varphi\left(\sum r_i m_i\right)+\sum r_j \varphi(e_j)
  \]
  con lo que \(\{m_i\}\cup\{\psi(e_j)\}\).
\end{proof}

Ejemplo de que no se puede mejorar la proposición anterior:
Sea \(I\) un conjunto infinito, \(K\) un cuerpo.
\[
  K^I=\{{(\alpha_i)}_i\in I:\alpha_i\in K\}
\]
\(K^I\) es un anillo finitamente generado
por \((\ldots,1,1,1,\ldots)\). Definimos:
\[
  K^{(I)}=\{{(\alpha_i)}_i\in I:\alpha_i\in K \textrm{ y } \alpha_i=0
  \textrm{ salvo un número finito de } i\in I\}
\]

Tenemos que \(K^{(I)}\) es un ideal de \(K^I\), y por tanto ideal a
izquierda, pero no es finitamente generado como ideal a izquierda.

Es decir, \(M\) finitamente generado no implica que un submódulo suyo
sea finitamente generado.

\begin{df}[Módulos Noetherianos]
  Un módulo finitamente generado \(M\) se dice Noetheriano si todo
  submódulo de \(M\) es finitamente generado.
\end{df}

El ejemplo anterior no era un módulo Noetheriano.

\begin{prop}
  Equivalen:
  \begin{enumerate}
    \item \(M\) es noetheriano.
    \item Cualquier cadena ascendente \(L_1\subseteq L_2\subseteq\ldots
      \subseteq L_n\subseteq\ldots\) se estabiliza, es decir,
      a partir de un cierto \(m\) las inclusiones se vuelven igualdades.
    \item Cada subconjunto no vacío de \(\mathcal{L}(M)\) tiene un elemento
      maximal con respecto de la inclusión.
  \end{enumerate}
\end{prop}

\begin{proof}
  Veamos que la primera implica la segunda.
  Tomamos:
  \[
    L=\bigcup_{n\ge 1} L_n\in\mathcal{L}(M)
  \]
  es un submódulo porque están encajados. Por hipótesis, es finitamente
  generado. Si tomamos un conjunto finito de generadores \(F\)
  tenemos que \(F\subset L\) y como es finito, debe existir un \(m\)
  suficientemente grande tal que \(F\subseteq L_m\) y como
  genera a \(F\) se tiene que \(L\subseteq L_m\subseteq L\)
  con lo que \(L_n=L_m=L\) para todo \(n\ge m\).

  Veamos que la segunda implica la primera. Sea \(\Gamma\subseteq
  \mathcal{L}(M)\) no vacío. Si \(\Gamma\) no tiene elemento maximal
  y tomamos \(L_1\in\Gamma\), entonces existe \(L_2\in\Gamma\)
  tal que \(L_1\subsetneq L_2\).

  Reiterando el proceso, tenemos que \(L_1\subsetneq L_2\subsetneq
  \ldots\subsetneq L_n\subsetneq\ldots\) no se estabiliza.

  Veamos que la tercera afirmación implica la primera.
  Sea \(N\in\mathcal{L}(M)\).
  Tomamos el conjunto \(\Gamma\) el conjunto de todos los submódulos
  finitamente generados de \(N\). Tenemos que el módulo trivial
  es finitamente generado, luego \(\Gamma\) es no vacío.

  Sea \(L\) un elemento maximal de \(\Gamma\). Veamos que \(L=N\).

  En caso contrario, tomamos \(x\in N\) tal que \(x\notin L\). Resulta que
  \(L+Ax\) es un submódulo de \(N\) y es finitamente generado.
  \(L+Ax\in\Gamma\) y \(L\neq L+Ax\), con lo que \(L\) no sería maximal.
\end{proof}

\begin{nt}
  \(N\in\mathcal{L}(M)\), escribimos \(N\le M\).
\end{nt}

\begin{prop}[Sucesiones exactas cortas en módulos noetherianos]
  Sea \(0\longrightarrow L\overset{\varphi}{\longrightarrow}
  M\overset{\psi}{\longrightarrow} N
  \longrightarrow 0\).

  Entonces \(M\) es noetheriano si y solo si \(L\) y \(N\) son
  noetherianos.
\end{prop}

\begin{proof}
  Supongamos \(M\) noetheriano.

  \(L\cong \Im\psi\le M\) y entonces \(L\) es noetheriano trivialmente.

  Tomamos \(N_1\subseteq N_2\subseteq\ldots\subseteq N_n\subseteq\ldots\)
  una cadena ascendente en \(\mathcal{L}(N)\).

  Tenemos \(\varphi^{-1}(N_1)\subseteq \varphi^{-1}(N_2)
  \subseteq \varphi^{-1}(N_n)\subseteq\ldots\)
  cadena en \(\mathcal{L}(M)\). Existe un \(m\) a partir del cual
  se estabiliza. Entonces, para todo \(n\ge n\):
  \[
    N_n=\varphi(\varphi^{-1}(N_n))=\varphi(\varphi^{-1}(N_m))=N_m
  \]
  con lo cual \(N\) es noetheriano.


  Supongamos ahora que \(N\) y \(L\) son noeherianos.
  Tomamos una cadena ascendente \(M_n\) de submódulos de \(M\).

  Por otro lado, \(M_n\cap\Im\psi\) es una cadena de submódulos de
  \(M\), que se estabiliza por ser noetheriano \(\Im\psi\cong L\).

  Tenemos \(\varphi(M_n)\) es una cadena de submódulos de \(N\),
  que también se estabiliza.

  Tomemos el menor natural tal que ambas cadenas se hayan estabilizado.
  Sea \(n\) mayor, \(x\in M_n\), \(\varphi(x)\in\varphi(M_n)
  =\varphi(M_m)\), debe existir \(y\in M_m\). Luego \(x-y\in\ker\varphi
  =\Im\psi\), con lo que \(x-y\in M_n\cap\Im\psi=M_m\cap\Im\psi\subseteq M_m\)
  y \(x\in M_m\) ya que \(y\in M_m\).

  Por tanto \(M\) es noetheriano.
\end{proof}

\begin{cor}
  Dados dos módulos \(M_1\) y \(M_2\).
  Entonces:
  \[
    M_1\oplus M_2
  \]
  es noetheriano si y solo si \(M_1\) y \(M_2\) lo son.
\end{cor}

\begin{proof}
  Sea la sucesión exacta corta
  \[
    0\longrightarrow M_1\longrightarrow M_1\oplus M_2\longrightarrow M_2
    \longrightarrow 0
  \]
  donde la primera aplicación es \(m_1\mapsto(m_1,0)\)
  y \((m_1,m_2)\mapsto m_2\) y el núcleo de la segunda es la imagen de
  la primera. Trivialmente se sigue el corolario.
\end{proof}

\begin{teo}
  Sea \(A\) un anillo. Cada módulo sobre \(A\) finitamente generado
  es noetheriano si y solo si \(\subscriptbefore{A}{A}\) es noetheriano.
\end{teo}

\begin{proof}
  Una de las implicaciones es obvia.

  Veamos que si el módulo regular es noetheriano, veamos que cualquier otro
  lo es.

  Sea \(M\) finitamente generado, existe un homomorfismo sobreyectivo \(\phi\)
  tal que \(A^n\longrightarrow M\).

  Usando inductivamente el corolario, tenemos que \(A^n\) es noetheriano.
  La proposición nos dice que \(M\) es noetheriano, aplicandolo
  a la sucesión
  \[
    0\longrightarrow\ker\phi\longrightarrow A^n\longrightarrow
    M\longrightarrow 0
  \]
\end{proof}

\begin{df}[Anillo noetheriano]
  \(A\) se dice noetheriano a izquierda si el módulo regular es
  noetheriano. Si \(A\) es conmutativo diremos simplemente noetheriano.
\end{df}

\begin{cor}
  Si \(A\) es noetheriano, equivalen para cualquier sucesión exacta corta:
  \begin{enumerate}
    \item \(M\) es finitamente generado.
    \item \(L\) y \(N\) son finitamente generados.
  \end{enumerate}
\end{cor}

\begin{cor}
  Todo dominio de ideales principales es noetheriano.
\end{cor}

\subsection{Módulo Artiniano}

\begin{df}[Módulo artinano]
  Para un \(\subscriptbefore{A}{M}\), son equivalentes:
  \begin{enumerate}
    \item Cada cadena descendente
      \(L_1\supseteq L_2\supseteq\ldots\supseteq L_n\supseteq\ldots\)
      de submódulos de \(M\)
      se estabiliza, esto es, a partir de cierto natural \(m\)
      se tiene \(L_n=L_m\) para todo \(n\ge m\).
    \item Cada subconjunto de \(\mathcal{L}(M)\) tiene un elemento
      minimal.
  \end{enumerate}
  A un tal módulo lo llamaremos artiniano.
\end{df}

Ejercicio: Sea \(A\) un dominio de integridad conmutativo. Si el
módulo regular es artiniano, entonces \(A\) es un cuerpo.

En particular \(\Z\) no es artiniano, aunque por ser un DIP, sí que
es noetheriano.

Ejercicio: \(K\) un cuerpo de característica 0. Tomo \(K[x]\) anillo
de polinomios. Veo \(K[x]\) como \(K\)-espacio vectorial.
Tomamos \(T\) la aplicación lineal \(T(f):=f'\), donde \(f'\) es el
polinomio derivado. Esto nos da una estructura de \(K[x]\)-módulo
sobre \(K[x]\) que no es la del módulo regular. Se pide demostrar
que ese módulo es artiniano y no finitamente generado.

En consecuencia, la estructura que hemos definido no es la misma
que la del módulo regular.

\begin{prop}
  Sea \[0\longrightarrow L\longrightarrow M\longrightarrow N\longrightarrow
  0\]

  Entonces \(M\) es artiniano si y solo si \(L\) y \(N\) son artinianos.
\end{prop}

Ejercicio: sea \(p\) un número primo. Definimos:
\[
  C_{p^\infty}=\{z\in\C:z^{p^n}=1\textrm{ para algún } n\ge1\}
\]
Se pide comprobar que es un subgrupo \(\S=\{z\in\C:\md{z}=1\}\) y demostrar
que visto como \(\Z\)-módulo es artiniano pero no es finitamente generado.



\end{document}
