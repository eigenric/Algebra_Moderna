\subsubsection{Módulos acotados sobre un DIP}

\begin{df}[Módulo acotado sobre un DIP]
  Sea \(A\) un dominio de ideales principales, \(\subscriptbefore{A}{M}\)
  un módulo, \(\Ann_A(M)=\langle\mu\rangle\) para cierto \(\mu\in A\).

  Si \(\mu\neq 0\), diré que \(M\) es acotado.
\end{df}

Supongamos que \(\subscriptbefore{A}{M}\) es acotado y \(\mu\not\in
\mathcal{U}(A)\), ya que si \(\mu\in\mathcal(A)\) entonces \(M=\{0\}\).

Si \(\mu=p_1^{e_1}\cdots p_t^{e_t}\), posible porque todo DIP es un
dominio de factorización única (DFU), con \(p_i\in A\) irreducible
y \(e_i>0\).

\begin{prop}[Descomposición primaria del módulo]
  Tomamos \(q_i=\frac{\mu}{p_i^{e_i}}\in A\).

  Llamamos \(M_i=\{q_i m:m \in M\}\subseteq M\). Veamos
  que \(M_i\in\mathcal{L}(\subscriptbefore{A}{M})=\{\textrm{submódulos
  de } \subscriptbefore{A}{M}\}\).

  Queremos que \(M=M_1\dot{+}\cdots \dot{+}M_t\), con \(t>1\) para
  evitar trivialidades. En ese caso, \(\mcd\{q_1,\ldots,q_t\}=1\),
  donde se ha usado que estamos en un DFU.\@

  Por la identidad de Bezout (válida porque estamos en un DIP),
  tenemos que \(1=\sum_{i=1}^t q_i a_i\), para ciertos \(q_i\in A\).
  Para en \(m\in M\), \(M=1\cdot m=\sum_i q_i a_i m\), luego
  \(M=M_1+\cdots+ M_t\).

  Vamos a ver que la suma es directa.
  \(q_i q_j\in\langle\mu\rangle\) si \(i\neq j\). Eso significa que si
  \(m\in M_i\) y entonces \(q_j m= 0\) si \(i\neq j\).
  Por tanto \(M_i=\{m\in N: m=q_i a_i m\}\).

  Si \(0=\sum_{i=1}^t\) con \(m_i\in M_i\), entonces
  \[
    0=q_j a_j 0=m_j
  \]
  y por tanto \(M=M_1\dot{+}\cdots\dot{+} M_t\).
\end{prop}

\begin{df}[Componentes primarias]
  Tenemos que los \(M_i\) se llaman componentes primarias.
\end{df}
\begin{prop}
  \[
    M_i=\{m\in M: p_i^{e_i} m=0\}
  \]

  Así, \(\langle\mu\rangle=\Ann_A(M)=\bigcap_{i=1}^t\Ann_A(M_i)
  \supseteq\bigcap_{i=1}^t\langle p_i^{e_i}\rangle=\langle\mu\rangle\)
\end{prop}

Ejercicio: Obtener la descomposición primaria usando \(\dot{+}\) de
\(\Z_{8000}\).
