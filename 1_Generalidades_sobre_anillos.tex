\section{Introducción}
\subsection{Generalidades sobre anillos}
\begin{df}[Anillo]
  Sea \(A\) un conjunto en el que existen dos operaciones
  \(+,\cdot:A\times A\longrightarrow A\) tales que:
  \begin{enumerate}
    \item \((A, +,0)\) es un grupo aditivo (conmutativo):
      \begin{itemize}
        \item \((a+b)+c=a+(b+c)\) para todos \(a,b,c\in A\).
        \item \(a+b=b+a\) para todos \(a,b\in A\).
        \item \(a+0=a\) para todo \(a\in A\).
        \item Para todo \(a\in A\) existe un \(-a\in A\)
          tal que \(-a+a=0\).
      \end{itemize}
    \item \((A, \cdot, 1)\) es un monoide:
      \begin{itemize}
        \item \((ab)c=a(bc)\) para todos \(a,b,c\in A\).
        \item \(a\cdot 1=1\cdot a=a\) para todo \(a\in A\).
      \end{itemize}
    \item Se cumplen las siguientes propiedades distributivas:
      \begin{itemize}
        \item \((a+b)c=ac+bc\) para todos \(a,b,c\in A\).
        \item \(a(b+c)=ab+ac\) para todos \(a,b,c\in A\).
      \end{itemize}
     \end{enumerate}
  \end{df}

  \begin{df}[Idelaes]
    Sea \(A\) un anillo. \(I\subset A\) se dice ideal si cumple las
    siguientes propiedades:
    \begin{itemize}
      \item \(I\) es un subgrupo aditivo de \(A\) (es decir,
        \(I\) es un conjunto no vacío que cumple
        \(b-a\in I\) para todo \(a, b\in I\)).
      \item \(ax, xa\in I\) para todo \(a\in I\) y \(x \in A\).
    \end{itemize}
  \end{df}

  \begin{teo}[Teorema de Isomorfía]
    Sea \(f:A\longrightarrow B\) un homomorfismo de anillos. Entonces:
    \begin{itemize}
      \item \(\ker f\) es un ideal de \(A\),
      \item \(\Im f\) es un subanillo de \(B\),
      \item Si \(I\subset \ker f\) es un ideal de \(A\), entonces
        existe un único homomorfismo de anillos tal que
        \(\tilde{f}:A/I\longrightarrow B\) tal que \(\tilde{f}(a+I)=f(a)\).
      \item El homomorfismo anterior es inyectivo si y solo si
        \(I=\ker f\).
      \item El homomorfismo anterior es sobreyectivo si y solo si lo era
        \(f\).
    \end{itemize}
  \end{teo}

