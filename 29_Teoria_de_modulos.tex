\begin{df}
\(A\) es un DIP, \(\subscriptbefore{A}{M}\) módulo.
\[
  t(M)=\{m\in M:\ann_A(m)\neq\langle 0\rangle\}
\]
es un submódulo de \(M\), que se llama submódulo de torsión de \(M\).
\end{df}


Ejemplo: sea \(A\) un DIP,
sea \(\subscriptbefore{A}{M}\) un módulo y consideramos
su submódulo de torsión.

Supongamos que \(t(M)\neq\{0\}\).
Definimos \(P\) como el conjunto de representantes de las clases de
equivalencia, bajo la relación ser asociados, de los irreducibles de \(A\).

Sea \(p\in P\), tomamos \(M_p=\{m\in M: p^e m=0\textrm{ para algún }
e\ge 1\}\). Tenemos que \(M_p\subseteq t(M)\), \(M_p\) es un submódulo.
Entonces:
\[
  t(M)=\dot{+}_{p\in P} M_p
\]
Demostremos esto.

Tomemos un \(m\in t(M)\), \(Am\) es un módulo de longitud finita.
\[
  Am=N_1\dot{+}\cdots\dot{+}N_r
\] donde \(N_i\) es una componente \(p_i\)-primaria.

En particular, \(m=m_1+\cdots+ m_r\) de manera que \(m_i\in N_i\subseteq
M_{p_i}\).

Luego \(M=\sum_{p\in P} M_p\). La unicidad es sencilla de deducir:
cada \(m\) estará en una componente primaria.

Caso particular. Tomamos \(M=\Cont^\infty(\R)\), \(M\) es un
\(\R[x]\)-módulo si \(xf=f'\). Entonces \(t(M)\) es el conjunto
de las funciones que satisfacen una EDO con coeficientes constantes.

\(P=\{\textrm{ Polinomios mónicos o bien lineales o bien
cuadráticos irreducibles}\}\). Es decir, cualquier función que se puede
definir mediante una EDO lineal con coeficientes constantes se puede
escribir como suma de funciones que resuelven
\({(\alpha\frac{\textrm{d}^2}{\textrm{d}x^2}+
\beta\frac{\textrm{d}}{\textrm{d}x}+\gamma)}^e f=0\) con \(e\in\N\).

Como hemos visto en ese caso particular, \(M_p\) no tiene por qué
tener longitud finita.

Consideremos \(I\) un conjunto infinito y \(R^{(I)}\) tal y como lo hemos
definido antes.
\begin{lema}
  Si \(M\) es un \(R\) módulo, existe una sucesión exacta de la forma
  \[
    0\longrightarrow L\longrightarrow R^{(I)}\longrightarrow M
    \longrightarrow 0
  \]
  para \(I\) adecuado.
\end{lema}
\begin{proof}
  Tomo \(\{m_i:i\in I\}\) tal que \(M=\sum_{i\in I}Rm_i\).
  Definimos \(\varphi:R^{(I)}\longrightarrow M\) dada por
  \(\varphi({(r_i)}_{i\in I})=\sum_{i\in I}r_i m_i\).

  \(L=\ker\varphi\overset{\iota}{\longrightarrow} M\).
\end{proof}

\begin{lema}[]
  Para \(\{m_i:i\in I\}\subseteq M\), son equivalentes:
  \begin{enumerate}
    \item \(\sum_{i\in I} r_i m_i=0\) implica que \(r_i=0\) para todo índice.
    \item El homomorfismo \(\varphi:R^{(I)}\longrightarrow M\) con
      \(\varphi({(r_i)}_{i\in I})=\sum_i r_i m_i\) es inyectiva.
  \end{enumerate}
  Si se satisface 1, diremos que el conjunto \(\{m_i:i\in I\}\) es linealmente
  independiente. Si además estos elementos son además un conjunto de
  generadores, diremos que forman una base.
\end{lema}

La demostración es trivial.

\begin{obs}
  \(M\) tiene una base si y solo si \(M\cong R^I\) para algún \(I\).
\end{obs}

\begin{df}[Módulos libres]
  Un módulo se llama libre si admite una base.
\end{df}

\begin{obs}
  Advertencia: hay muchos módulos que no son libres.
\end{obs}

Ejemplos de módulos no libres:
\begin{enumerate}
  \item Ningún grupo abeliano finito es libre como \(\Z\) módulo.
  \item \(t(M)\), \(\subscriptbefore{A}{M}\) con \(A\) un DIP, nunca es libre.
    En otras palabras \(A^{(I)}\) no es nunca un módulo de torsión (por
    ser un dominio de integridad).
\end{enumerate}
