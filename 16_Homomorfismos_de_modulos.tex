\begin{df}[Base de un módulo libre]
  Consideramos \(A^n=A\oplus\cdots\oplus A\), donde la suma se repite
  \(n\) veces. Para cada \(i=1,\ldots, n\), tenemos que
  \(\{e_i: e_i=(0,\ldots, 0,1,0, \ldots, 0)\}\) forman un sistema de
  generadores de \(A^n\). Por tanto \(a=\sum_i a_i e_i\in A^n\)
  es una expresión única.
\end{df}

Dicha base puede no existir.

\begin{prop}
  Dado un módulo cualquiera \(\subscriptbefore{A}{M}\) y \(m_1, m_n\in M\),
  existe un único homomorfismo de módulos \(f:A^n\longrightarrow M\)
  tal que \(f(e_i)=m_i\).
\end{prop}

\begin{cor}
  Si \(M\) es finitamente generado con generadores \(\{m_i\}\),
  entonces \(M\cong A^n/L\) para \(L\) un sierto submódulo.
\end{cor}

\begin{proof}
  Unicidad: si existe una tal aplicación \(f\), entonces para
  cualquier \(a\in A^n\),
  \[
    f(a)=\sum_i a_i f(e_i)=\sum_i a_i m_i
  \]

  Veamos la existencia,
  Definiendo \(f(a)=\sum_i a_i m_i\) obtenemos un homomorfismo de módulos
  que cumple lo exigido en el enunciado.

  Si \(M=Am_1+\cdots+Am_n\) tenemos que \(L=\ker f\) cumple lo que se
  pide por el teorema de isomorfía para módulos.
\end{proof}

