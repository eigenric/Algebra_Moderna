\begin{df}
  \[
    \Ring(R,S)=\{\phi:R\longrightarrow S, \phi \textrm{ es homomorfismo
    de anillos}\}
  \]
\end{df}


\begin{prop}
  Dados un grupo aditivo \(M\) y un anillo \(A\), se tiene una
  correspondencia biyectiva entre:
  \begin{enumerate}
    \item Homomorfismos de anillos \(\rho:A\longrightarrow\End(M)\)
    \item Las aplicaciones \(A\times M\longrightarrow M\)
      que satisfacen:
      \begin{itemize}
        \item \((a+a')m=am+a'm\)
        \item \(a(m+m')=am+am'\)
        \item \((aa')m=a(a'm)\)
        \item \(1\cdot m=m\)
      \end{itemize}
  \end{enumerate}
\end{prop}

\begin{proof}
  Tomamos la biyección \(\psi^{-1}:\Map(A,\Map(M,M))\longrightarrow
  \Map(A\times M,M)\).
  Tomamos \(\rho\in\Ring(A,\End(M))\), su imagen por la biyección,
  \(\psi^{-1}(\rho)\)
  son las aplicaciones que satisfacen justo las propiedades
  anteriores.

  Llamamos a \(\psi^{-1}(\rho)(a,m)=a\cdot m\). Tenemos que
  \(\psi^{-1}(\rho)(a,m)=\rho(a)(m)\). Entonces
  \(a\cdot m=\rho(a)(m)\).

  Comprobamos la tercera propiedad como ejemplo:

  Dados \(a, a'\in A\) y \(m\in M\):
  \[
    (aa')m=\rho(aa')(m)=(\rho(a)\circ\rho(a'))(m)
    =\rho(a)(\rho(a')(m))=\rho(a)(a'm)=a(a'm)
  \]

  De forma análoga se demuestran el resto de propiedades.

  Esta correspondencia responde a la fórmula \(am=\rho(a)(m)\).
\end{proof}

Un \(A\)-módulo lo veré de cualquiera de las maneras anteriores, que
ya hemos visto que son equivalentes, según su conveniencia.

Ejemplo, si \(K\) es un cuerpo, un \(K\)-módulo es esencialmente
un \(K\) espacio vectorial.

Otro ejemplo, el \(A\)-módulo regular. \(A\) es un \(A\)-módulo, vía
\(\lambda:A\longrightarrow(A)\) que lleva cada
\(a\) a \(\lambda(a)(a'):=aa'\). La demostración es sencilla
usando la segunda definición.

\begin{prop}[Restricción de escalares]
  Sea \(\phi:R\longrightarrow S\) homomorfismo de anillos. Si \(M\)
  es un \(S\)-módulo, vía un homomorfismo de anillos \(\rho:
  S\longrightarrow\End(M)\), tenemos que \(M\) es un
  \(R\)-módulo vía \(\rho\circ\phi\).

  Equivalentemente, si \(r\in R\) y \(m\in M\), definimos
  \[
    rm=(\rho\circ\phi)(r)(m)=\rho(\phi(r))(m)=\phi(r)m
  \]
\end{prop}

