\subsubsection{Módulos abstractos}

Sea \(A\) un anillo, \(M_A\) un \(A\)-módulo,
entonces si tenemos
un homomorfismo de anillos \(\varphi:A\longrightarrow\End(M)\)
cuyo núcleo es un ideal de \(A\).

Aplicando el primer teorema de isomorfía, tenemos:
\[
  A/\ker\varphi\longrightarrow\Im\varphi\subseteq\End(M)
\]
y entonces \(M\) es un \(A/\ker\varphi\)-módulo.
De hecho \((a+\ker\varphi)m=\varphi(m)\).

\[
  \ker\varphi=\{a\in A: am=0\}=\Ann_A(M)
\]
se le llama el anulador de \(M\).

Tenemos que \(M_A\) entonces \(M_{A/\Ann_A(M)}\)
