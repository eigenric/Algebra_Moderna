\subsection{Módulos abstractos}

Sea \(A\) un anillo, \(\subscriptbefore{A}{M}\) un \(A\)-módulo,
entonces si tenemos
un homomorfismo de anillos \(\varphi:A\longrightarrow\End(M)\)
cuyo núcleo es un ideal de \(A\).

Aplicando el primer teorema de isomorfía, tenemos:
\[
  A/\ker\varphi\longrightarrow\Im\varphi\subseteq\End(M)
\]
y entonces \(M\) es un \(A/\ker\varphi\)-módulo.
De hecho \((a+\ker\varphi)m=\varphi(m)\).

\[
  \ker\varphi=\{a\in A: am=0\}=\Ann_A(M)
\]
se le llama el anulador de \(M\).

Tenemos que \(\subscriptbefore{A}{M}\) entonces \(M_{A/\Ann_A(M)}\)

Ejercicio: si tenemos una plicación lineal entre espacios vectoriales
de dimensión finita, entonces el anulador está generado por un único
polinomio, el polinomio mínimo de \(T\).

\begin{df}
  Un submódulo de un módulo
  \(\subscriptbefore{A}{M}\) es un subgrupo aditivo \(N\subseteq M\)
  tal que \(am\in N\) para cualquier \(a\in A\) y \(m\in N\). Los
  submódulos del módulo regular \(A\) se llaman ideales por la izquierda
  de \(A\).
\end{df}

\begin{obs}
  Todo ideal es un ideal a izquierda. Si \(A\) es conmutativo, los ideales
  a izquierda coinciden con los ideales.
\end{obs}

Ejemplo: tomando \(A=\mathcal{M}_2(K)\) con \(K\) un cuerpo.
\[
  \mathcal{M}_2(K)=\left\{
    \begin{pmatrix}
      a&b\\
      c&d
    \end{pmatrix}:
    a,b,c,d\in K
  \right\}
\]

Tenemos que el conjunto:
\[
  \left\{
    \begin{pmatrix}
      0&b\\
      0&d
    \end{pmatrix}:
    b,d\in K
  \right\}
\]
es un ideal a izquierda de \(A\).

Ejemplo: \(T:V\longrightarrow V\), \(K\)-lineal.
¿Qué es un \(K[x]\)-submódulo de \(V_{K[x]}\)?
Sea \(W\) un tal submódulo.
\(W\) es un subespacio vectorial y además \(T(w)=xw\in W\),
es decir, un subespacio \(T\)-invariante (un ejemplo
de subespacio \(T\) invariante es un subespacio propio).
El recíproco es también cierto.

\begin{df}[Submódulo cíclico]
  Dado \(\subscriptbefore{A}{M}\), y un \(m\in M\). Es claro que
  \(Am=\{am:a\in A\}\) es un submódulo de \(\subscriptbefore{A}{M}\) que se llama submódulo
  cíclico generado por \(m\).
\end{df}

Ejemplo: \(\R[x]\sin t=\R\sin t+\R\cos t\).

\begin{df}[Submódulo finitamente generado]
  Dados \(m_1,\ldots, m_n\in M\), el conjunto
  \[
    Am_1+\cdots+Am_n=\{a_1 m_1+\cdots+a_n m_n: a_i\in A\}
  \]
  es un submódulo de \(\subscriptbefore{A}{M}\) llamado el submódulo generado por
  \(m_1,\ldots, m_n\).
  Si \(M=Am_1+\cdots+Am_n\), diremos que \(M\) es finitamente generado
  con generadores
  \(m_1,\ldots, m_n\).
\end{df}

\subsubsection{Suma directa interna}

\begin{df}[Módulo suma]
  Dados \(N_1,\ldots, N_n\) submódulos de \(\subscriptbefore{A}{M}\), defino:
  \[
    N_1+\cdots+ N_n=\{m_1+\cdots+m_n: m_i\in N_i\}
  \]
  es un submódulo de \(M\) que se llama suma de \(N_1+\cdots+N_n\).
\end{df}

\begin{nt}
  Se puede expresar \(N_1+\cdots+ N_n\) como \(\sum_{i=1}^n N_i\).
\end{nt}

\begin{prop}
  Sean \(N_1,\ldots, N_t\) submódulos de \(A\). Son equivalentes:
  \begin{enumerate}
    \item \(N_i\cap\sum_{j\neq i}N_j =\{0\}\) para todo \(i\).
    \item Si \(0=n_1+\cdots+n_t\), \(n_i\in N_i\) entonces \(n_i=0\)
      para todo \(i\).
    \item Cada \(n\in N_1+\cdots+N_t\) admite una representación
      única como \(n=n_1+\cdots+n_t\) con \(n_i\in N_i\).
  \end{enumerate}
\end{prop}
