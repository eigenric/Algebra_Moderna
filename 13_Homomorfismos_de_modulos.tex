\subsection{Homomorfismos de módulos}

\begin{df}[Módulo cociente o factor]
  Sea \(\subscriptbefore{A}{M}\) y \(L\in\mathcal{L}(M)\).
  Consideramos \(M/L\) grupo aditivo y se define la acción:
  \[
    a(m+L):=am+L
  \]
  \(M/L\) es un módulo.
\end{df}

\begin{df}[Homomorfismo de módulos]
  Se dice que
  \(f:\subscriptbefore{A}{M}\longrightarrow \subscriptbefore{A}{N}\)
  es un homomorfismo de módulos si respeta sumas y productos.
\end{df}

\begin{df}[Proyección canónica]
  Es la aplicación \(\pi:M\longrightarrow M/L\) dada por
  \(\pi(m)=m+L\) es un homomorfismo de módulos.
\end{df}

\begin{teo}[Teorema de isomorfía para módulos]
  \(f:M\longrightarrow N\) un homorfismo de \(A\)-módulos. Entonces
  el núcleo \(\ker f\in\mathcal{L}(\subscriptbefore{A}{M})\) y
  \(\Im f\in\mathcal{L}(N)\). Para cada \(L\in\mathcal{L}
  (\subscriptbefore{A}{M})\) tal que \(L\subseteq \ker f\) existe
  un único homomorfismo de módulos \(\tilde{f}:M/L\longrightarrow N\)
  tal que \(\tilde{f}\circ\pi=f\). Finalmente, \(\tilde{f}\) es
  inyectiva si y solo si \(L=\ker f\), en cuyo caso, \(\tilde{f}\)
  da un isomorfismo de \(A\)-módulos
  \(M/\ker f\cong \Im f\).
\end{teo}

Ejemplo \(\subscriptbefore{A}{M}\), definimos \(f:A\longrightarrow M\)
dada por:
\[
  f(a)=am\hspace{1cm} \forall a \in A
\]
es un homomorfismo de \(A\)-módulos.

Tenemos \(\Im f = Am\) y
\(\ann(a)=\ker f=\{a\in A: am=0\}\) es un ideal izquierda y se tiene
\[
  A/\ann_A(m)\cong Am
\]

