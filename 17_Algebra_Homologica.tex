\section{Módulos Noetherianos}
\subsection{Álgebra homológica}

\begin{df}[Sucesiones exactas]
  Una suceión de homomorfismos de módulos \(f_i:M_i\longrightarrow
  M_{i+1}\) se dice exacta en
  \(M_{i+1}\) si \(\ker f_{i+1}=\Im f_i\).
\end{df}

Ejemplo: Dada una sucesión \(\{0\}\longrightarrow L \alpha
\longrightarrow M \beta\longrightarrow N\longrightarrow \{0\}\)
es exacta en \(L\) si y solo si \(\ker \alpha=\{0\}\), es decir,
\(\alpha\) es inyectiva, en \(N\) si y solo si \(\Im \beta = N\),
es decir, \(\beta\) sobreyectiva y en \(M\) si y solo si
\(\ker\beta=\Im\alpha\).

A \(\alpha\) se les llama monomorfismos de módulos y a
\(\beta\) epimorfismos de módulos.

A esta sucesión se le llama sucesión exacta corta.

Caso particular: Por ejemplo, si \(f:M\longrightarrow N\) es
un homorfismo de módulos, obtenemos:
\[
  0\longrightarrow\ker f\iota\longrightarrow M f\longrightarrow\Im f
  \longrightarrow 0
\]

\begin{prop}
  Sea \(0\longrightarrow L\overset{\psi}{\longrightarrow} M
  \overset{\varphi}{\longrightarrow}
  N\longrightarrow 0\) una sucesión exacta de \(A\)-módulos. Entonces:
  \begin{enumerate}
    \item Si \(M\) es finitamente generado, lo es también \(N\).
    \item Si \(L\) y \(N\) son finitamente generados, lo es también \(M\).
  \end{enumerate}
\end{prop}

\begin{proof}
  Veamos primero la primera afirmación. Sea \(\{m_i\}\) generadores de \(M\).
  Es claro que \(\{\varphi(m_i)\}\) generan \(N\).

  Para la segunda, \(\{n_i\}\) generadores de \(N\), y tomamos
  \(\{m_i\}\subseteq M\) tales que \(\varphi(m_i)= n_i\).

  Tomamos \(\{e_i\}\) generadores de \(L\). Tomamos \(m\in M\).
  \[
    \varphi(m)=\sum_{i=1}^s r_i n_i = \sum_{i=1} r_i\varphi(m_i)
    =\varphi\left(\sum r_i m_i\right)
  \]
  con lo que \(m-\varphi(\sum r_i m_i)\in\ker\varphi=\Im\psi\).
  Luego existen \(b_1,\ldots, b_t\) tales que
  \[
    m-\varphi\left(\sum r_i m_i\right)=
    \psi\left(\sum_j b_j e_j\right)
  \]
  y finalmente:
  \[
    m=\varphi\left(\sum r_i m_i\right)+\sum r_j \varphi(e_j)
  \]
  con lo que \(\{m_i\}\cup\{\psi(e_j)\}\).
\end{proof}

Ejemplo de que no se puede mejorar la proposición anterior:
Sea \(I\) un conjunto infinito, \(K\) un cuerpo.
\[
  K^I=\{{(\alpha_i)}_i\in I:\alpha_i\in K\}
\]
\(K^I\) es un anillo finitamente generado
por \((\ldots,1,1,1,\ldots)\). Definimos:
\[
  K^{(I)}=\{{(\alpha_i)}_i\in I:\alpha_i\in K \textrm{ y } \alpha_i=0
  \textrm{ salvo un número finito de } i\in I\}
\]

Tenemos que \(K^{(I)}\) es un ideal de \(K^I\), y por tanto ideal a
izquierda, pero no es finitamente generado como ideal a izquierda.

Es decir, \(M\) finitamente generado no implica que un submódulo suyo
sea finitamente generado.

\begin{df}[Módulos Noetherianos]
  Un módulo finitamente generado \(M\) se dice Noetheriano si todo
  submódulo de \(M\) es finitamente generado.
\end{df}

El ejemplo anterior no era un módulo Noetheriano.

\begin{prop}
  Equivalen:
  \begin{enumerate}
    \item \(M\) es noetheriano.
    \item Cualquier cadena ascendente \(L_1\subseteq L_2\subseteq\ldots
      \subseteq L_n\subseteq\ldots\) se estabiliza, es decir,
      a partir de un cierto \(m\) las inclusiones se vuelven igualdades.
    \item Cada subconjunto no vacío de \(\mathcal{L}(M)\) tiene un elemento
      maximal con respecto de la inclusión.
  \end{enumerate}
\end{prop}
