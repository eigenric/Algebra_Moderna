\subsection{Módulo Artiniano}

\begin{df}[Módulo artinano]
  Para un \(\subscriptbefore{A}{M}\), son equivalentes:
  \begin{enumerate}
    \item Cada cadena descendente
      \(L_1\supseteq L_2\supseteq\ldots\supseteq L_n\supseteq\ldots\)
      de submódulos de \(M\)
      se estabiliza, esto es, a partir de cierto natural \(m\)
      se tiene \(L_n=L_m\) para todo \(n\ge m\).
    \item Cada subconjunto de \(\mathcal{L}(M)\) tiene un elemento
      minimal.
  \end{enumerate}
  A un tal módulo lo llamaremos artiniano.
\end{df}

Ejercicio: Sea \(A\) un dominio de integridad conmutativo. Si el
módulo regular es artiniano, entonces \(A\) es un cuerpo.

En particular \(\Z\) no es artiniano, aunque por ser un DIP, sí que
es noetheriano.

Ejercicio: \(K\) un cuerpo de característica 0. Tomo \(K[x]\) anillo
de polinomios. Veo \(K[x]\) como \(K\)-espacio vectorial.
Tomamos \(T\) la aplicación lineal \(T(f):=f'\), donde \(f'\) es el
polinomio derivado. Esto nos da una estructura de \(K[x]\)-módulo
sobre \(K[x]\) que no es la del módulo regular. Se pide demostrar
que ese módulo es artiniano y no finitamente generado.

En consecuencia, la estructura que hemos definido no es la misma
que la del módulo regular.

\begin{prop}
  Sea \[0\longrightarrow L\longrightarrow M\longrightarrow N\longrightarrow
  0\]

  Entonces \(M\) es artiniano si y solo si \(L\) y \(N\) son artinianos.
\end{prop}

Ejercicio: sea \(p\) un número primo. Definimos:
\[
  C_{p^\infty}=\{z\in\C:z^{p^n}=1\textrm{ para algún } n\ge1\}
\]
Se pide comprobar que es un subgrupo \(\S=\{z\in\C:\md{z}=1\}\) y demostrar
que visto como \(\Z\)-módulo es artiniano pero no es finitamente generado.

