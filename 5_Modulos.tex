\section{Módulos}

\begin{df}
  Sean \(M\), \(N\) grupos aditivos:
  \[
    \Ad(M,N)=\{f:M\longrightarrow N| f\textrm{ homomorfismo de grupos}\}
  \]
\end{df}

El conjunto anterior es no vacío porque \(0\in\Ad(M,N)\).
\(\Ad(M,N)\) es un grupo aditivo con la suma:
\[
  (f+g)(m):=f(m)+g(m)\hspace{1cm} \forall m\in M
\]


\begin{df}[Anillo de endomorfismo de \(M\)]
  Definimos directamente \(\End(M):= \Ad(M,M)\).
\end{df}

\begin{prop}
  \((\End(M),+,0,\circ,\id) \) es un anillo.
\end{prop}
\begin{proof}
  Se comprueba que es cerrado para composición. Es obvio que la
  composición es asociativa y tiene como elemento neutro la identidad.

  Finalmente se ve que se cumplen las propiedades distributivas, que
  se siguen de que son homomorfismos.
\end{proof}


\begin{obs}
  Consideramos el grupo \(\{0\}\), es el anillo \(\{0\}\) (anillo
  cero o trivial).

  Si \(M\neq \{0\}\), entonces \(\End(M)\) no es trivial.
\end{obs}

\begin{df}[Módulo]
  Sea \(M\) un grupo aditivo y \(A\) un anillo. Una estructura de
  \(A\)-módulo sobre \(M\) es un homomorfismo de anillos
  \(\rho: A\longrightarrow\End(M)\).
\end{df}

Ejemplo: los números enteros. \(M\) grupo aditivo, \(A=\Z\).
Existe un único \(\chi:\Z\longrightarrow\End(M)\) determinado
por \(\chi(1)=\id_M\), es decir, una única estructura de
\(\Z\)-módulo sobre \(M\) (y su núcleo te da la
característica del anillo).

Ejemplo: cuerpos. Sea \(K\) un cuerpo.
Si \(V\) es un \(K\)-espacio vectorial, definimos \(\rho:K\longrightarrow
\End(V)\), tomamos \(\rho(\alpha):V\longrightarrow V\)
cumpliendo \(\rho(\alpha)(v)=\alpha v\). Trivialmente se cumple que
\(\rho\) es un homomorfismo por la estructura de espacio vectorial de \(V\).
Con lo cual tenemos una estructura de \(K\)-módulo sobre \(V\).
Se puede demostrar el recíproco trivialmente.

\begin{obs}
  Sean  \(X, Y, Z\) conjuntos. \(\Map(X,Y)\) es el conjunto de
  aplicaciones de \(X\) en \(Y\).

  Entonces:
  \[
    \psi:\Map(X\times Y, Z)\longrightarrow\Map(X,\Map(Y,Z))
  \]
  es una biyección dada por \(\psi(f)(x)(y):=f(x,y)\) y
  \(\psi^{-1}(g)(x,y):=g(x)(y)\).
\end{obs}

Ejercicio: comprobar que \(\psi^{-1}\) es realmente la inversa de
\(\psi\).

\begin{obs}
  Sean \(M, N, L\) grupos aditivos.
  \[
    \psi:\Biad(M\times N, L)\longrightarrow\Ad(M,\Ad(N,L))
  \]
  donde \(b\in\Biad(M\times N, L)\) si \(b\) es biaditiva:
  \[
    b(m+m',n)=b(m,n)+b(m',n)
  \]\[
    b(m,n+n')=b(m,n)+b(m,n')
  \]
\end{obs}

Ejercicio, demostrar que la aplicación \(\psi\) es una biyección.

\begin{teo}[Caracterización de módulos]
  Sea \(A\) anillo, \(M\) un grupo aditivo. Sea \(\Ring(A,\End(M))\),
  llamamos \(A\)-módulo a la imagen por \(\psi\) de ese conjunto.
\end{teo}
